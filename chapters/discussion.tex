\chapter{Discussion}
This chapter expand on and discuss the presented findings, compare it with the reviewed literature and state its relevance to the research questions. The research objectives was to…. explore users experience and perception of the digital nudge trial, and gain information about how aspects like framing and delivery method affects the perception and experience. 

By answering the RQs we are able to gather insights that could be used when tailoring digital nudge interventions in the future. 

\section{Overskrift?}

\subsection{Automatic vs. reflective processing}
Caraban et al. \cite{caraban_23_2019} stated that the automatic systems should be further exploited when it comes to technology for influencing behaviour, as most of our choices are made by this system, however few studies seem to approach it. The choice to engage in physical activity, such as workout or take a walk is usually not an automatic choice as it is conceivable that for many people physical activity is perceived as more time-consuming behavior than, for example, remember to use dental floss or wash your hands properly. This means that this behavior often requires planning before the implementation can actually happen. However, when physical activity becomes a habit, the choices are made more automatically (people have a knack for acquiring routines). This means that the choice of physical activity can be a result of both automatic processing (I go to training today because I do every Monday) and reflective processing (I go to training today because it's good for me). It seems that the implemented nudge intervention tap onto both the automatic system (when the push notification catches attention and is read) and the reflective system (when participants reflects on the presented information). However, it is only when the reflective systems is touched that the nudge seem to have a positive effect(?).But what our findings indicate may be that for receive users attention, we approach the automatic system, though the push notification. And this could mean?

\subsection{Educational effect}
One would believe that most people are aware of the conveyed health information information as it is consensus and not high level knowledge, and obtained from actors such as WHO, FHI and Helsedir. Our findings show that some of the information (especially regarding cancer) was new to the participants. The fact that some participants actually said they had expanded their reasoning and motivation to be physically active suggests that it has an educational effect. Norwegian citizens know a lot (as most of the consensus information was recognized) but there is still a lot of information that has not reached everyone yet. The digital nudging may not have had an immediate effect on choices and behaviors as such, but it did have an effect on attitudes and knowledge, which often underlies behavior. Further, this proves that there is a need to make information from such actors more accessible, for instance through digital nudging (related to mobile apps). 

\subsection{Technology channel/delivery method}
It is known that the first step in persuasion is getting users attention, which could be inherited to digital nudges in the form of signals as well. The DND method suggest choosing an appropriate technology channel, but there is a lack of research on which channels are most beneficial for different types of nudging.  All we know is that there are many to choose from. We chose to implement digital nudging through push notification because it was an easily accessible feature and because it ensures that as many as possible are reached without seeming intrusive. Our findings confirm this assumption?? Furthermore, it was interesting to compare it with other delivery methods, e.g. Helsedir SMS nudging. 

\subsection{Covid-19 Nudging: communication channel and source credibility}
As stated in Chapter 5, the covid-19 pandemic gave us the opportunity to compare our nudging intervention with a recent case of governmental use of nudging. Helsedirektoratets nudging did not address physical activity promotion, but it conveyed other types of recommendations on behavior related to public health, more specifically regarding infection control. The two are similar considering that they both promote health preventive behaviors, and are virtually based on the same source (as the health information in Sats intervention was endorsed by Helsedirektoratet and FHI). The main difference is that the information was broadcasted through different channels, and our findings indicates that they are perceived differently in regards of attention, commitment and severity/urgency. Utilizing a more personal communication channel, SMS was experienced as more direct than push notification, with the intention that is was read and absorbed more or less instantly by all receivers. Participants told that they almost felt like a civic duty. There was no room for reactions other than to follow the recommendations, i.e. they felt commitment to engaging in the behaviour. Whereas the push notifications also felt like receiving recommendations, they expressed that it was up to each individual to figure out how to respond or act on it. 

Further, it seems that SMS has greater certainty of reaching the user, as push notifications often get lost in the information flow. Push notification captures the user's attention and is read at some point, but the information is not experienced as seriously as the SMS from Helsedir. 

What is more important; the message itself or the communication channel? It seems that the choice of channel could have greater impact than the actual content. 

How would it be perceived if Helsedir sent health information about inert (?) and slow moving disease, motivated by the gap between recommended activity levels and the actual activity levels. Could it be that such communication could obscure and dilute more urgent messages? Would the public approve of it? And would it strengthen the sense of commitment if Sats had actually stated that the information was advices from public actors such as the Helsedir. 

These findings indicates how the sender or source is crucial for perception of message. This can be compared to the work by authors X \cite{oinas-kukkonen_persuasive_2009} (Reference: Persuasive Systems Design: Key Issues, Process Model, and System Features) where it is stated that the credibility is an important property of a system aiming to persuade the user about a behaviour. The sender of the nudges (i.e. Sats) is perceived as trustworthy and domain expert by most of participants. However, Helsedir is perceived as trustworthy to an even greater extent and therefore has greater influence. It is conceivable that the perception of seriousness and sense of commitment would be strengthened if it was stated that the information originally comes from Helsedir. 

\subsection{Gain vs. loss}
Our findings indicate that in the context of physical activity promotion amongst gym-members, gain-framed nudges are perceived as more effective than loss-framed nudges. This is consistent with previous literature on message framing in other physical activity related context. 

\subsection{Ethics of nudging}
The fact that the participants understood the intention of nudging and that they felt that their freedom of choice was maintained proves that the implemented intervention is in line with the nudging requirements.

\subsection{Psychological effects}
Biases and heuristics are depend on each other, hence it was not just the framing effect that emerged through the findings. From Self-determination theory we know that humans are in need of feeling autonomy in the context of exercising behavior. I.e. if technology aiming to influence decision-making could imitate that feeling it is more likely to be successful and achieve that choice or behaviour. Our findings show that participants experience the choice of engaging in physical activity to be truly their own, even though they were exposed to external influences. 

It seems that participants that expressed that loss-framed nudges was most effective felt cognitive dissonance; they are a member of the gym with an intention to have good health. When negative effects are presented people will do everything to ensure that this does not apply to them, because that would feel uncomfortable when they in fact have the intention to be at good health. 

\subsection{Minimal Effect}
Some participants pointed out that they did not think the digital nudging had any effect on them (does not mean that they experienced it negatively, but more indifferent). One of the reasons for this could be because they considered the effect to be an immediate action or choice regarding workout after receiving the nudge. Hence, as they did not decide to or plan a workout immediately after reading the message they concluded that there was no effect. However, the aim to influence people's choices regarding behaviour should more be seen as a process than a single choice and its effect. This means that such interventions need to be investigated over a long period of time in order to say something about the real effect. As we wanted to understand how the intervention was perceived, effect was not limited to the degree of achieved immediate behavior, i.e. that the nudge led to a workout or not. This should probably have been made clearer during the interview. (it doesn't need to be a workout at Sats, all kind of physical activity is good).

\subsection{When nudges fail: Counter effects}
Caraban et al. \cite{caraban_23_2019} made us aware of the possibility of negative effects of nudging, and emphasize how this should be taken into account in understanding how we can design effective nudges. Our findings show that some nudges led to counter effects for a minor part of the participants. In particular, the loss-framed nudges were experienced as patronizing, and hence demotivating, for some participants. This implies that some nudges actually leads to the opposite of desired behaviour. Others viewed the nudge intervention as a whole as moralizing and provocative, considering its content as the focus on illnesses and disorders. There was apparently no connection (no common characteristics) between the users that voiced the digital nudging intervention as a whole and those who voiced loss-framed nudges, as impacting in a negative way. What these participants had in common was that they did not exercise for their health, they trained because it was fun (expect the guy that was demotivated, he was not interested in training at all). This is not at all favorable, and needs to be considered for future applications. Such effects are not mapped by quantitative studies, which implies the importance of this qualitative assessment.  


\section{RQ1: How are digital nudges based on consensus health information, experienced and perceived by gym members?}

The study shows two distinct directions among participants / gym members perception and experience of the intervention. 

A larger group of participants experienced positive effect in the form of reminding, affirmative and uplifting. 


On the other hand, a smaller group of participants felt that it was moralizing to communicate health information in this way. These participants had particular opinions about the loss-framed nudges. 

However, we found no apparent correlation in personal characteristics between the participants who expressed the different paths, and further studies on more specific user groups among this target population must be done in order to establish more concrete guidelines for tailoring. 
\section{RQ2: How does message framing affect users perception of the digital nudge?}

It should be said that the basis for exploring message framing was amputated with regard to Covid-19, so the features are few and weak.

There has been a lot of focus on message framing up over the years with persuasive research to influence behavior such as physical activity. It has been pointed out that the framing of a message has great impact.  

Strong and negative opinions about loss-framed nudges were expressed, while neutral (and positive) opinions about gain-framed. Even though this only holds for a minor part of the participants, this points out that using the wrong framing can have a greater negative effect than the effect that comes from using the framing correctly. I.e. when applying message framing to digital nudging or similar persuasive interventions, it is crucial to avoid using wrong framing. To be on the safe side, practitioners implementing digital nudge interventions should therefore frame their message from gain perspective, until otherwise is proven.  We also tried to find correlations between users characteristics (defined user groups) in regards of preferred message framing, but did not find any of significance. 

Greater impact in the negative direction if incorrect framing is used, than impact in the right direction, i.e. message framing is not the ..

On the other side, it was found that message framing did not affect users perception of digital nudge intervention to great extent, and other factors seemed to be of greater impact. 


\section{RQ3: How are push notifications experienced as the channel for broadcasting digital nudges?}

Push notifications are experienced as a suitable way of digital nudging. 
Push notifications makes information available / easily accessible but also easy ignorable (hence does not interuptive)  
Effortless but also commitmentless (due to content)

The nudges could in principle be broadcasted in different ways, for instance through SMS, email or in app notification. The advantage of push notifications is that they are widely read, compared to eg mail.

Furthermore, it appears that the sender plays a role in users perceiving the information as credible.

\section{Limitations}
There are some limitations regarding this study that needs to be addressed.

\subsection{Changes and adjustments in regards of Covid-19 }
The plan was to broadcast the digital nudge intervention over a 30 day period, which is originally very short in terms of influencing behavior. Due to the covid-19 pandemic it was not possible to complete the intervention, and only 8/16 nudges were sent out. This makes the basis to test holistic health information limited and it is difficult to spot standout nudges. Further, this weakens the findings, and additional research needs to be conducted in order to confirm the preliminary findings of this study. There is also a educational aspect - For instance we found that the conveyed holistic health information, most of it was known from before but something proves to be new. By testing the nudging intervention over a longer period of time (preferably over several months) one would gain a better understanding of what information is catching, being remembered, new, etc for the participants. The holistic picture of this is needed to say something specific about tailoring digital nudges. 

It is also conceivable that doing both pre and post interviews would strengthen the understanding, as we could see how attitudes, knowledge, intentions, motivation and behavior change after the intervention. Similarly, quantitative data can help to understand the bigger picture.

This also led to some of our findings not being based on real-world perception and experience anyway, as we had to present some of the nudges during the interview.

To adapt to the circumstances and still be able to answer the research questions,, we chose to present some of the nudges during the interview. This also led to some of our findings (especially those concerning framing) not being based on real-world perception and experience, as first intended. 

\subsection{Sample selection / Variations across participants}
As it was an exploratory research we found it okay to invest a bigger group. This was also done in regards of convenience sampling. However it turned out that many persons wanted to participate in the study, and we could have been more selective when scheduled interviews. 

Due to this, there is a variation across characteristics such as age, motivation type, prior physical activity level/habits, etc. 

To be on the safe side, we gathered quantitative data about participants in order to be able to define groups within the sample, but we did not see any connections / correlations between the characteristics (in addition it turned out to be out of scope for this thesis). 
Due to time constrictions, corona and ethics of doing research we used convenience sampling. As we imposed convenience sampling for this study, the 
Among those who signed up as participants in the study, the selection was representative. But due to the time constrictions (lost time due to corona) we had to make interviews with the ones available first. We did not have time to sort of age and gender, even though we had the opportunity for it in regards of number of participants and the variation in age and gender. The time limitations was one of the consequences of corona pandemic. 

The intention to include training members was that among this group / selection of people there will always be someone who trains a lot and someone that does not, meaning / in other words; it was representative for a large part of Norway's population. However, all members of training center share / has one ting in common; the motivation / intention to engage in some form of physical activity, because they already became a member (invest money in their own health). When we consider Fogg's behaviour model, we see that motivation is one of three fundamental elements that needs to be in place for a behaviour to occur. Ability and prompts are the other two. 

\subsection{Definitions and interpretations }
During the interview we clearly stated how we defined physical activity, ie.e that it did not have to concern a workout at Sats. However, participants often tend to use the term workout or training. For instance “It did not motivate me to workout”. Which makes some of the details about the data somewhat ambiguous. This is something I should have made clearer during the interview: it doesn't need to be a workout at Sats, all kind of physical activity is good.
The same regards the term perceived effect, which can be defined in many ways. The question regarding their perceived effects was open and without restrictions, in order to capture the different types of real and honest effect.


