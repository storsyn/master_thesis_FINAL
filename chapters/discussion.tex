\chapter{Discussion}
This chapter expand on and discuss the presented findings, and compare it with findings of the previous studies reviewed in chapter 2. 



\subsection{Automatic vs. reflective processing}

\subsection{Educational effect}
One would believe that most people are aware of the conveyed health information information as it is consensus and not high level knowledge, and obtained from actors such as WHO, FHI and Helsedir. Our findings show that some of the information (especially regarding cancer) was new to the participants. The fact that some participants actually said they had expanded their reasoning and motivation to be physically active suggests that it has an educational effect. Norwegian citizens know a lot (as most of the consensus information was recognized) but there is still a lot of information that has not reached everyone yet. The digital nudging may not have had an immediate effect on choices and behaviors as such, but it did have an effect on attitudes and knowledge, which often underlies behavior. Further, this proves that there is a need to make information from such actors more accessible, for instance through digital nudging (related to mobile apps). 

\subsection{Covid-19 Nudging: communication channel and source credibility}
As stated in Chapter 5, the covid-19 pandemic gave us the opportunity to compare our nudging intervention with a recent case of governmental use of nudging. Helsedirektoratets nudging did not address physical activity promotion, but it conveyed other types of recommendations on behavior related to public health, more specifically regarding infection control. The two are similar considering that they both promote health preventive behaviors, and are virtually based on the same source (as the health information in Sats intervention was endorsed by Helsedirektoratet and FHI). The main difference is that the information was broadcasted through different channels, and our findings indicates that they are perceived differently in regards of attention, commitment and severity/urgency. Utilizing a more personal communication channel, SMS was experienced as more direct than push notification, with the intention that is was read and absorbed more or less instantly by all receivers. Participants told that they almost felt like a civic duty. There was no room for reactions other than to follow the recommendations, i.e. they felt commitment to engaging in the behaviour. Whereas the push notifications also felt like receiving recommendations, they expressed that it was up to each individual to figure out how to respond or act on it. 

Further, it seems that SMS has greater certainty of reaching the user, as push notifications often get lost in the information flow. Push notification captures the user's attention and is read at some point, but the information is not experienced as seriously as the SMS from Helsedir. 

What is more important; the message itself or the communication channel? It seems that the choice of channel could have greater impact than the actual content. 

How would it be perceived if Helsedir sent health information about inert (?) and slow moving disease, motivated by the gap between recommended activity levels and the actual activity levels. Could it be that such communication could obscure and dilute more urgent messages? Would the public approve of it? And would it strengthen the sense of commitment if Sats had actually stated that the information was advices from public actors such as the Helsedir. 

These findings indicates how the sender or source is crucial for perception of message. This can be compared to the work by authors X (Reference: Persuasive Systems Design: Key Issues, Process Model, and System Features) where it is stated that the credibility is an important property of a system aiming to persuade the user about a behaviour. The sender of the nudges (i.e. Sats) is perceived as trustworthy and domain expert by most of participants. However, Helsedir is perceived as trustworthy to an even greater extent and therefore has greater influence. It is conceivable that the perception of seriousness and sense of commitment would be strengthened if it was stated that the information originally comes from Helsedir. 

\subsection{Gain vs. loss}
Our findings indicate that in the context of physical activity promotion amongst gym-members, gain-framed nudges are perceived as more effective than loss-framed nudges. This is consistent with previous literature on message framing in other physical activity related context. 

\subsection{Ethics of nudging}
The fact that the participants understood the intention of nudging and that they felt that their freedom of choice was maintained proves that the implemented intervention is in line with the nudging requirements.

\subsection{Psychological effects}
Biases and heuristics are depend on each other, hence it was not just the framing effect that emerged through the findings. From Self-determination theory we know that humans are in need of feeling autonomy in the context of exercising behavior. I.e. if technology aiming to influence decision-making could imitate that feeling it is more likely to be successful and achieve that choice or behaviour. Our findings show that participants experience the choice of engaging in physical activity to be truly their own, even though they were exposed to external influences. 

It seems that participants that expressed that loss-framed nudges was most effective felt cognitive dissonance; they are a member of the gym with an intention to have good health. When negative effects are presented people will do everything to ensure that this does not apply to them, because that would feel uncomfortable when they in fact have the intention to be at good health. 

\subsection{Minimal Effect}
Some participants pointed out that they did not think the digital nudging had any effect on them (does not mean that they experienced it negatively, but more indifferent). One of the reasons for this could be because they considered the effect to be an immediate action or choice regarding workout after receiving the nudge. Hence, as they did not decide to or plan a workout immediately after reading the message they concluded that there was no effect. However, the aim to influence people's choices regarding behaviour should more be seen as a process than a single choice and its effect. This means that such interventions need to be investigated over a long period of time in order to say something about the real effect. As we wanted to understand how the intervention was perceived, effect was not limited to the degree of achieved immediate behavior, i.e. that the nudge led to a workout or not. This should probably have been made clearer during the interview. (it doesn't need to be a workout at Sats, all kind of physical activity is good).

\subsection{When nudges fail: Counter effects}
Caraban et al. (reference) made us aware of the possibility of failing nudges, and emphasize how this should be taken into account in understanding how we can design effective nudges. Failing nudges, does not refer to nudges that did not successfully achieve desired behaviour but rather to backfiring effects. Our findings some that some nudges led to negative / encounter effects for a minor part of the participants. In particular, the loss-framed nudges were experienced as patronizing, and hence demotivating, for some participants (with extrinsic motivation). This implies that some nudges actually leads to the opposite of desired behaviour. Others (with intrinsic motivation) viewed the nudge intervention as a whole as moralizing and provocative, considering its content as the focus on illnesses and disorders. There was apparently no connection (no common characteristics) between the users that voiced the digital nudging intervention as a whole and those who voiced loss-framed nudges, as impacting in a negative way. What these participants had in common was that they did not exercise for their health, they trained because it was fun (expect the guy that was demotivated, he was not interested in training at all). This is not at all favorable, and needs to be considered for future applications. Such effects are not mapped by quantitative studies, which implies the importance of this qualitative assessment.  


\section{RQ1: How are digital nudges based on consensus health information, experienced and perceived by gym members?}



\section{RQ2: How does message framing affect users perception of the digital nudge?}

\section{RQ3: How are push notifications experienced as the channel for broadcasting digital nudges?}


\section{Additional investigations triggered by Covid-19}

%Future work eller discussion? HVOR SKAL DETTE 
Even though this way of nudging (informational health nudges) immediately invites the reflective mind, it would be interesting to see how such intervention would lead to over time.  For future work, it would also be interesting to look at different types and ways of implementation. In particular looking at those who talks to the automatic mind. 

\section{Limitations}
There are some limitations regarding this study that needs to be addressed.

\subsection{Changes and adjustments in regards of Covid-19 }

\subsection{Sample selection / Variations across participants}
Due to time constrictions, corona and ethics of doing research we used convenience sampling. As we imposed convenience sampling for this study, the 

Among those who signed up as participants in the study, the selection was representative. But due to the time constrictions (lost time due to corona) we had to make interviews with the ones available first. We did not have time to sort of age and gender, even though we had the opportunity for it in regards of number of participants and the variation in age and gender. The time limitations was one of the consequences of corona pandemic. 

The intention to include training members was that among this group / selection of people there will always be someone who trains a lot and someone that does not, meaning / in other words; it was representative for a large part of Norway's population. However, all members of training center share / has one ting in common; the motivation / intention to engage in some form of physical activity, because they already became a member (invest money in their own health). When we consider Fogg's behaviour model, we see that motivation is one of three fundamental elements that needs to be in place for a behaviour to occur. Ability and prompts are the other two. 

\subsection{Definitions and understandings}

