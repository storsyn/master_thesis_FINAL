\chapter{Discussion}
The research objectives for this study was were to investigate users perception and experiences of the implemented digital nudge intervention (among others), which was successfully fulfilled, except from the intervention that was not able to complete due to Covid-19. However, valuable qualitative data have been gathered, sorted, analysed and will now be discussed. Firstly, this chapter expands on and discusses the presented findings and compare it with the reviewed literature. A separate part will state its relevance towards the specific research questions that this thesis address. To the end, a separate part will state its relevance towards the specific research questions that this thesis address.  

\bigbreak
\textbf{RQ1: How does message framing affect the user perception of the digital nudge?}
\bigbreak
This research questions was mainly inspired by two studies; 1) the fact that we don't know how the framing effect will be transferred to digital nudging \cite{mirsch_digital_2017}, 2) And that framing should continue to be researched in for new and broaden context concerning physical activity 
\cite{williams_effects_2019}. Hence, studying the use of message framing in a digital nudge intervention for physical activity promotion (through push notification and mobile application). 
\bigbreak
\textbf{RQ2: How are push notifications experienced as the channel for broadcasting digital nudges?}
\bigbreak
Several studies and statements implies that digital nudging often occurs in the form of push notifications, and also not intended in many cases (referring to that digital nudging can often be observed without the designers behind the implementation necessarily following the nudging principles consciously \cite{mirsch_making_2018}), which motivate this research question regarding push notification as a future platform for conscious implementation of digital nudging. This research question is addressed by the full scale implementation where push notification is qualitatively examined for this particular usage and context. The users attitude and experiences have been studied, also, by putting it in the relative perspective to other technology channels. 
\bigbreak
\textbf{RQ3: How are digital nudges based on consensus health information, experienced and perceived by gym members? }
\bigbreak
The use of health information to motivate, increase, support and maintain physical activity have been studied many times, but not in the digital nudging context, and for gym members. This is an wide question revolving around the overall experience and perception of the implemented digital nudge intervention, where feelings, engagement depth, potential effects, intentions, interests, PA habits, previous experiences, motivation, attitude, etc, are all considered in order to answer the question. 

By answering these RQs we are able to gather insights that could be used when tailoring digital nudge interventions for PA in the future. 

\section{General Discussion}

\subsection{Automatic vs. reflective processing}
Caraban et al. \cite{caraban_23_2019}stated how the automatic processing system should be further exploited when it comes to technological interfaces for influencing behaviour, as most of our choices are made by this system. However, Meske and Amojo \cite{meske_status_2019} pinpoint how the core of nudging should be, and is, more about cognitive thinking and reflection. The choice to engage in physical activity, such as workout or go for a walk is usually not an automatic choice. Many people consider such behaviour time-consuming, much more so than other nudging contexts, e.g. remember to use dental floss or wash your hands properly. This means that PA behavior often requires planning before execution. However, when physical activity becomes a habit, the choices are made more automatically (people have a knack for acquiring routines). This means that the choice of physical activity can be a result of both automatic processing (I go to training today because I do every Monday) and reflective processing (I go to training today because it's good for my health). It seems that the implemented nudge intervention tap onto both the automatic system (when the push notification catches attention and is read) and the reflective system is mobilized (when participants reflects on the presented information). 

The perception process could be viewed as a timeline; starting point (attention + automatic) and ending point (reflective + effect). Some participants terminates in the first phase and do not reach the reflective phase,  thus they have only read the information without devoting more thoughts. For this group, the idea of PA stuck to mind through the day. For the other group of participants that passed this phase and ended up in the reflective phase, this yielded more valid and positive effects, i.e. affirmative, uplifting, rewarding, etc. Meaning that the nudging intervention had been effective both when automatic processing and reflective processing was mobilized, but to different degrees. Automatic: neutral opinions towards the effect. Reflective: positive affect / positive opinion, perceptions and experiences to the effect. 

%This way: this context: training app and sats), which is very positive  because then the nudge has worked as intended in terms of stimulating t physical activity. The eduvational effect shows that the digital nudge intervention was succesful 

%\subsection{Delivery method}
%It is known that the first step in persuasion is getting users attention, which could be inherited to digital nudges in the form of signals as well. The DND method \cite{mirsch_making_2018} suggest choosing an appropriate technology channel, but there is a lack of research on which channels are most beneficial for different types of nudging.  All we know is that there are many to choose from. We chose to implement digital nudging through push notification because it was an easily accessible feature and because it ensures that as many as possible are reached without seeming intrusive. Our findings confirm this assumption. Furthermore, it was interesting to compare it with other delivery methods, e.g. Helsedir SMS nudging. 

\subsection{Communication channel and source credibility}
It is known that the first step in persuasion is getting users attention, which could be inherited to digital nudges in the form of signals as well. The DND method \cite{mirsch_making_2018} suggest choosing an appropriate technology channel, but there is a lack of research on which channels are most beneficial for different types of nudging.  All we know is that there are many to choose from. We chose to implement digital nudging through push notification because it was an easily accessible feature and because it ensures that as many as possible are reached without seeming intrusive. Our findings confirm this assumption. Furthermore, it was interesting to compare it with other delivery methods, e.g. Helsedir SMS nudging. 


As stated in Chapter 5, the Covid-19 pandemic gave us the opportunity to compare our nudging intervention with a recent case of governmental use of nudging. Helsedirektoratets nudging did not address physical activity promotion, but it conveyed other types of recommendations on behavior related to public health, more specifically regarding infection control. The two are similar considering that they both promote health preventive behaviors, and are virtually based on the same source (as the health information in Sats intervention is largely endorsed by Helsedirektoratet and FHI). The main difference is that the information was broadcasted through different channels (and by different sources), and our findings indicates that they are perceived differently with regards to attention, commitment and severity. Utilizing a more personal communication channel, SMS was experienced as more direct than push notification, with the intention that it was read and absorbed more or less instantly by all receivers. Participants told that they almost felt like a civil duty. There was no room for reactions other than to follow the recommendations, i.e. they felt commitment to engaging in the behaviour (both due to source credibility and use of channel). Whereas the push notifications also felt like receiving recommendations, they expressed that it was up to each individual to figure out how to respond or act on it. Further, it seems that SMS has greater certainty of reaching the user, as push notifications often get lost in the information flow. Push notification captures the user's attention and is read at some point, but the information is not experienced as seriously as the SMS from Helsedir. What is more important; the message itself or the communication channel? It seems that the choice of channel could have greater impact than the actual content. How would it be perceived if Helsedirektoratet broadcasted health information about inert and slow moving diseases, motivated by the gap between recommended activity levels and the actual activity levels. Could it be that such communication could obscure and dilute more urgent messages? Would the public approve of it? It is conceivable that the sense of commitment would be strengthen if Sats had actually stated that the information was advices from authorities such as the Helsedirektoratet. 

These findings indicates how the sender and source is crucial for perception of message. This can be compared to the work by Oinas-Kukkonen and Harjumaa\cite{oinas-kukkonen_persuasive_2009}
where it is stated that the credibility is an important property of a system aiming to persuade the user about a behaviour. The sender of the nudges (i.e. Sats) is perceived as trustworthy and domain expert by most participants. However, Helsedirektoratet is perceived as trustworthy to an even greater extent and therefore had greater influence. It is conceivable that the perception of seriousness and sense of commitment would be strengthened if it was stated that the information originally came from Helsedirektoratet. 

%Gain vs loss
%Our findings indicate that in the context of physical activity promotion amongst gym-members, gain-framed nudges are perceived as more effective than loss-framed nudges. This is consistent with previous literature on message framing in other physical activity related context and the prospect theory. 

%Further it was also found that the participants had stronger emotions regarding the loss-framed nudge, than gain-framed, which is consistent with the loss aversion theory ????? (Reference). 

%Skal følgende avsnitt være her under RQ svar eller på General Discussion?

\subsection{Framing and Content}
Our findings indicate that in the context of physical activity promotion amongst gym-members, gain-framed nudges are perceived as more effective than loss-framed nudges. This is consistent with previous literature on message framing in other physical activity related context and the prospect theory \cite{williams_effects_2019}.
%(reference: The effects of message framing characteristics on physical activity education: A systematic review ). 
Further it was also found that the participants had stronger emotions regarding the loss-framed nudge, than gain-framed, which is consistent with the loss aversion theory. 

The message framing was not an obvious property to the participants. It is probable that this would have become more recognize if the implementation lasted for longer and with more repetitions (i.e. rolling out same message again and/or rolling out positive/negative alternative of the same message). People were generally more positive to positive framing. Many stated that the messages were new information or served as reminders. The impact the messages made were to a large degree conditional as in relatable. Younger participants had a tendency to displace health risk information. 

%The fact that the participants understood the intention of nudging and that they felt that their freedom of choice was maintained proves that the implemented intervention is in line with the nudging requirements. 

%In addition, it's hard to distinguish between digital nudging and persuasive design. In fact, it seems a bit paradoxical that nudging is based on protecting users' freedom of choice and should appear as transparent when at the same time attempt at influencing behaviours through automatic and unconscious processing. Seems like an intention to distancing from manipulation. (Reference: Status Quo, Critical Reflection, and the Road Ahead of Digital)

\subsection{Effects}
\subsubsection{Educational effect}
One would believe that most people are aware of the conveyed health information as it is consensus and not expert level knowledge, and obtained from organizations such as WHO, FHI and Helsedir. Our findings show that some of the information (especially regarding cancer) was new to the participants. The fact that some participants actually said they had expanded their reasoning and motivation to be physically active suggests that it had an educational effect. Participants were quite knowledgeable as most of the consensus information was recognized, but there is still a lot of information that has not reached the wider audience. The digital nudging may not have had an immediate effect on choices and behaviors as such, but it did have an effect on attitudes and knowledge, which often precedes real-world choices. Further, this proves that there is a potential to make information from such sources more accessible, for instance through digital nudging (related to mobile apps). 

\subsubsection{Psychological effects}
Biases and heuristics are depend on each other, hence it was not just the framing effect that emerged through the findings. From Self-determination theory we know that humans are in need of feeling autonomy in the context of exercising behavior\cite{orji_persuasive_2018}. I.e. if technology aiming to influence decision-making could imitate that feeling it is more likely to be successful and achieve that choice or behaviour. Our findings show that participants experience the choice of engaging in physical activity to be truly their own, even though they were exposed to external influences. 

It seems that participants that expressed that loss-framed nudges was most effective felt cognitive dissonance; they are a member of the gym with an intention to have good health. When negative effects are presented people will do everything to ensure that this does not apply to them, because that would feel uncomfortable when they in fact have the intention to be at good health. 

\subsubsection{Hidden effects}
Some participants pointed out that they did not think the digital nudging had any effect on them. This does not mean that they experienced it negatively, but they were simply indifferent. One of the reasons could be that they considered the effect to be an immediate action or choice regarding workout after receiving the nudge. Hence, as they did not decide to or plan a workout immediately after reading the message they concluded that there was no effect. However, the aim to influence people's behaviors should be regarded more as a long term prevailing process rather than a single opt-in/out choices and its effect. This means that such interventions need to be investigated over a long period of time in order to say something about the real effect. As we wanted to understand how the intervention was perceived, effect was not limited to the degree of achieved immediate behavior, i.e. that the nudge led to a workout or not. This could probably have been made clearer during the interview.

\subsubsection{Counter effects}
Caraban et al. \cite{caraban_23_2019} made us aware of the possibility of negative effects of nudging, and emphasize how this should be taken into account in understanding of designing effective nudges. Our findings show that some nudges led to counter effects for a minor part of the participants. In particular, the loss-framed nudges were experienced as patronizing, and hence demotivating, for some participants. This implies that this nudges actually lead to the opposite of desired behaviour. Others viewed the nudge intervention as a whole as moralizing and provocative, considering its content being focused on illnesses and disorders. There was apparently no connection (no common characteristics) between the users that voiced negative impact. This is not at all favorable, and needs to be considered for future applications. Previous quantitative studies have not been able to detect such effects, which supports the importance of this qualitative assessment.   

\subsection{Validity of nudges}
It could be argued if the digital nudge intervention differs from similar implementations of persuasive technology, such as reminders and prompts. Digital nudging serve as a more gentle way to influence people's decision-making opposed to persuasive tech/persuasion. The latter tends to be experienced as more aggressive and demanding \cite{meske_status_2019}. 

Participants experiences are compliant with the mere definition of a nudge, despite the fact that there is a commercial player behind the implemented digital nudge intervention. For instance, participants experienced that their freedom to choice was safeguarded, which, according to Thaler and Sunstein \cite{thaler_nudge-_2009} is an important property to be defined as a nudge. Also, the participants clearly understood the intention of the nudging and there was no hidden agenda, i.e. the nudges were perceived as transparent. It is an important feature of a nudge that is not manipulative\cite{karlsen_recommendations_2019}. In addition, the interaction is expected to come as a long term effect of the prevailing influence of the nudges and mainly after contemplation. This supports the principle of raising awareness and increasing knowledge in order to influences decision-making. The majority of participants seemed to appreciate that the digital nudge did not require direct interaction with the user interface. It was experienced as a free choice to read and act on the information, and they could relate to it as they wanted. The flipside would be if the nudge required some kind of digital interaction and response from the user, e.g. a link to book a particular training class.

\section{Research Questions}
\subsection{RQ1: How are digital nudges based on consensus health information, experienced and perceived by gym members?}

The study shows two distinct directions among participants perception and experience of the intervention; food for thought and patronizing. A larger group of participants experienced positive effect in the form of reminding, affirmative, infinitesimal motivation and uplifting. On the other hand, a smaller group of participants felt that it was moralizing to communicate health information with focus on risks in this manner. These participants had particular strong opinions about the loss-framed nudges that they found rude, tops-down and intruding.  

The contemplative and the reflective psychological effect can be found in many parts of the analysed data. Many stated that some of the information was new to them and that it may have started reflective and/or subconscious awareness accordingly. Other information serve as reminders, and had a particular impact when user could relate to it (as in tying it to real-life experiences). Younger participants, not surprisingly, seemed to relate less to messages because of displacement. This coincide with the general perception that the predominant factor w.r.t. health risk is in any case age.

However, we found no apparent correlation in personal characteristics between the participants who expressed the different opinion paths, and further studies on more specific user groups among this target population should be done in order to establish more concrete guidelines for tailoring. 

\subsection{RQ2: How does message framing affect users perception of the digital nudge?}

It should be noted that the basis for exploring message framing was cut short because of Covid-19 and the national lock-down, thus some of the results have to been assessed in lieu. The message framing was not apparent to participants, until it was displayed as a direct comparison during the interview. Those who responded the strongest were those who reacted negatively. However the majority of participants expressed a neutral perception towards the gain and loss framed nudges. It could be the case that the framing was not pinpointed sufficiently and powerful enough, and in that way was perceived neither positive or negative. By following the DND method to the full extent, and performing several iterations, one could enhance the framing and strengthen this contrast. One could envision using more explicit contents, e.g. graphics and visualizations to evoke stronger and more immediate reactions in either direction. 

Strong and negative opinions about loss-framed nudges were expressed, whilst opinions about gain framed were predominantly neutral or slightly positive. Even though this was only true for a minor part of the participants, using the wrong framing can have a greater negative effect than the effect that comes from using the framing correctly. I.e. when applying message framing to digital nudging or similar persuasive interventions, it is crucial to avoid using wrong framing. To be on the safe side, practitioners implementing digital nudge interventions should therefore frame their message from gain perspective, until otherwise is proven. We also tried to find correlations between users characteristics (defined user groups) in regards of preferred message framing, but did not find any of significance. 

\subsection{RQ3: How are push notifications experienced as the channel for broadcasting digital nudges?}
Utilizing push notification as channel for broadcasting have a (strong) impact on how users perceive and experience the digital nudging. First of all, it makes information available and comprehensible, which is good. The message reaches the target audience effectively and it requires nothing from the users.

On the other side, that the information is presented as a notification, i.e. one-way communication and that the user interface neither encourages digital interaction nor requires anything of the participant (other than reflective thinking and real world choices/behaviour), make them both easy to ignore and forget. Since the digital nudging intervention is a feature of an already existing app (legacy system), the effect/experience/perception may have been enhanced if the nudge specifically encouraged further interaction with the app, where participants also were exposed to other elements of persuasive technology (and digital nudge mechanism) (e.g. prior behaviour data). As literature suggests, digital nudging could be regarded as a subspace of persuasion and they complement each other.  

\section{Limitations}
There are some limitations regarding this study that needs to be addressed.

\subsection{Changes and adjustments in regards of Covid-19 }
The plan was to broadcast the digital nudge intervention over a 30 day period, which is originally very short in terms of influencing behavior. Due to the covid-19 pandemic it was not possible to complete the intervention, and only 8/16 nudges were sent out. This made the basis to test holistic health information limited and it was difficult to spot standout nudges. Further, this weakens the findings, and additional research needs to be conducted in order to confirm the preliminary findings of this study. 

To adapt to the circumstances and still be able to answer the research questions, we chose to present some of the nudges during the interview. This also led to some of our findings (especially those concerning framing) not being based on real-world perception and experience, as first intended. 

By testing the nudging intervention over a longer period of time (preferably over several months) and more repetition one would gain a better understanding of what information is catching, being remembered, new, etc for the participants. The holistic picture of this is needed to say something specific about tailoring digital nudges. It is also conceivable that doing both pre and post interviews would strengthen the understanding, as we could see how attitudes, knowledge, intentions, motivation and behavior change after the intervention. Similarly, quantitative data can help to understand the bigger picture. 

%There is also a educational aspect. For instance we found that the conveyed holistic health information, most of it was known from before but something proves to be new. 

\subsection{Sample selection}
As it was an exploratory research we found it acceptable to investigate a bigger group of participants. There was an element of over recruitment in order to secure enough interviewees. Due to the confusion related to the national lock down and uncertainties related to this, a significant delay was incurred (it was at the time uncertain if the intervention could be completed). When the situation had settled the interview sessions had to be scheduled on a convenience basis, first come first serve. Due to this, the intended sample selection had to be discarded and a random group was opted for instead. This explains the variation across characteristics such as age, motivation type, prior physical activity level, etc. In a randomly selected group of gym members there will be a variation in training level and frequencies. And because further delineation was not possible within the time frame, this became the accepted alternative. We gathered some quantitative meta data about participants in order to be able to define groups within the sample. However, all members of training centers have one ting in common; the motivation and intention to engage in some form of physical activity, because they already became a paying member. When we consider Fogg's behaviour model, we see that motivation is one of three fundamental elements that needs to be in place for a behaviour to occur (ability and prompts are the other two). 

We did not find many tendencies, i.e. as in overlapping opinions, within the defined user groups. The few correlations that were established is noted in the presented findings. 


\subsection{Definitions and interpretations }
During the interview we clearly stated how we defined physical activity, i.e. that it did not have to concern a workout at Sats. However, participants often tend to use the term workout or training. For instance “It did not motivate me to workout”. Which makes some of the details about the data somewhat ambiguous. This is something we should have made clearer during the interview: it doesn't need to be a workout at Sats, all kind of physical activity would be interesting to map. The same concerns the term perceived effect, which can be defined in many ways. The question regarding their perceived effects was open and without restrictions, in order to capture the different types of real and honest opinions.


