\chapter{Conclusion} 
%Final remarks, concluding thoughts, and suggestions for further research.

\section{Final remarks}

\section{Concluding thoughts}
This study aimed at investigating how a digital nudge intervention for physical activity promotion was received, perceived and experienced by users at a gym-center. By doing this, we gathered insights which contributes to the bigger question on how to nudge.

Ble heller en generell innnsiktsamling på hvordan oppleveseln av slike nudger er og potensiale. 

Generelt sett har vi samlet masse innsikt på hvordan digital nudging burde bli implementert for fsysisk aktivtet. Dette er første steg mot å forstå mer om hvordan vi kan designe effektive nudger for denne konteksten. 

Ved å inkludere brukere og kvalitative data har vi greid å ... 

Nyttig innsikt har blitt avslørt ved å se på message framing in the context of digital nudging, 

Det kan tenkes at dette er et funn som kan gjelde for andre persuasive technologier også, men det kreves videre forskning for å bekrefte dette. 

What did we find out:
- Gain is preferred 
- Using push notification as the method for delivery has both strength and weaks, more interaction and commitment could make it more promising
- Examples of Self-determination theory appeared in the study, indicating that this should be taken into consideration for next implementations
- Dette er en lovende bruk at digitale nudger (så lenge du ikke bruker loss på folk som ikke vil ha det)
- Som vi allerede vet er timing og context viktig, 
- Other types of content, more concrete and suggesting etc (facilitator)

Findings suggest that commitment 

By doing a qualitative assessment of the digital nudging intervention, we obtained some opinions that do not appear in a quantitative study. For example, we have found that loss-framed can cause negative / opposite effects for some users, which is highly undesirable. 

For å si noe mer spesifikt om tailoring, må det også gjøres bedre kategoriseringer på folk på treningssenter. 

By making a qualitative assessment of the nudging intervention, we obtain some opinions that do not appear in a quantitative study. For example, we have found that loss-framed can cause negative effects for some users

\section{Further research}
Ut fra funnene ser vi at det kan være interresant å benytte seg av annen type nudging enn (signal). Facilitator burde bli undersøkt ettersom brukere understerker at de trenger noe mer konkret, ikke bare helseinformasjon. 
Det ville også være interresant å kombinere informasjonen med forslag, for å oppmuntre til både interaksjon med appen. Direkte kommunkasjon. gjøre dette - kommando. 

Annet type innhold, annen leveranse, annen mekanisme og bygge på andre biases og heuristics. 
Sammenligne ulike implementeringer for å finne ut hva som er det beste for denne konteksten. 

Future research should continue to test and evaluate various digital nudging implementations in real-world context, so that we can increase our knowledge of how to nudge through digital environments. As well as how we can make the best use of technology and human psychology to help and support good choices in relation to health. 

Digital nudging involves many components; mechanisms, what psychological effects should we utilize, delivery methods etc that work best for the given context. This study was the first attempt to evaluate a digital nudge intervention based on qualitative data, and has contributed a lot of new insights that cannot be revealed with quantitative studies.

It would be interesting to do similar user experience studies concerning other types of digital nudges, to get a insights from the users perspectives. Of course, it could strengthen the findings if the study was rearranged but with controlled user groups, including both inactive and active participants.   

Researchers should continue to do qualitative studies around users' experiences and perceptions of various digital nudge implementations. 

By applying the framing effect to digital nudging for physical activity we also saw other psychological effects... 

There is a need to explore other psychological effects that can be applied in digital nudging interventions, to make them more effective.

In future, researchers should use the design frameworks and guidelines, in order to implement digital nudge interventions, 

Even though this way of nudging (informational health nudges) primarily aims at the reflective mind, it would be interesting to see how such intervention would lead to over time.  For future work, it would also be interesting to look at different types and ways of implementation. In particular looking at those who talks to the automatic mind. 

%Hva hvis våre helsenudger hadde blitt broadcasted til hele norges befolkning? vi antar det ville blitt oppfatte i det negative leiet. 


\subsection{Long-term effects}
It is known that studies regarding the long term effects is low for many HCI topics, including digital nudging. It would be extremely interesting to study this context in longer time interval, to see if this informational nudges had an impact over time. One of the participants actually expressed that "maybe over time, it would have a stronger effect on me". 