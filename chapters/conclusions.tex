\chapter{Conclusion} 
This chapter aims to.....

\section{Concluding thoughts}
This study aimed at investigating how a digital nudge intervention for physical activity promotion was received, perceived and experienced by users at a gym-center. By doing this, we gathered insights which contributes to the bigger question on how to nudge.

To the best of our knowledge, this study is one of the first attempts in investigating an actual implemented digital nudging intervention with a qualitative approach. It also strengthens HCI's role in this branch of the research field. Although the research on technology in the physical activity promotion domain has been extensive, this serves as the first study that defines/approaches intentional digital nudging within this context (could exist observational examples of digital nudging and PA). 

The technical solution worked as intended and it was possible to retrofit the push notifications to the existing app, there was no doubt about who was the sender. The sender Sats, being a commercial player, still was perceived as a credible source and even though the nudges evoked different feelings, there were few questions about the credibility of the contents.

Findings from the study could be important in order to map and establish more specific design guidelines for this context in the future. However, generalizations should not be made on the basis of the data, since a larger amount, and quantitative data is needed.

Although the project has encountered challenges, especially with regard to Covid-19 and the associated complications, changes and delays, it has met all objectives. By performing a qualitative assessment of the digital nudging intervention, we obtained important opinions that is not possible to identify in a quantitative study. The addressed research questions were successfully covered by data analysis and findings. For example, we have found that loss-framed nudges can cause negative effects for some users, which is highly undesirable. 

The contemplative and the reflective psychological effect can be found in many parts of the analysed data. Many stated that some of the information was new to them and that it may have started reflective and/or subconscious awareness accordingly. Other information serve as reminders, and had a particular impact when user could relate to it (as in tying it to real-life experiences). Younger participants, not surprisingly, seem to relate less to messages because of displacement. This coincide with the general perception that the predominant factor w.r.t. health risk is in any case age. 

Another important finding was that some experienced educational effect, which underpinned and expanded their reasoning for staying physical active. It is likely that this could lead to increased training frequency accordingly. 

%In a full rollout, i.e. more messages and more repetition, this effect could be confirmed and possibly strengthen. 

The digital nudging intervention was successful because the messages, and the opinions expressed, falls within the definition of nudging ; it should be experienced as a free choice, easy to opt out, transparent and should not enforcing. Participants appreciated that the nudges came without commitment. This was confirmed by the majority of the participants. 

A foundation for full scale nudge scheme is in place, and it is likely to have a quantitative effect in the long perspective. 

%The message framing was not an obvious property to the participants. It is probable that this would have become more recognize if the implementation lasted for longer and with more repetitions (i.e. rolling out same message again and/or rolling out positive/negative alternative of the same message). People were generally more positive to positive framing. Many stated that the messages were new information or served as reminders. The impact the messages made were to a large degree conditional as in relatable. Younger participants had a tendency to displace health risk information. 


\section{Further research}

If more time and resources were available this study could have been expanded on. 
For example by doing several iterations of the design phase, based on gathered user insights.  

Pre-post interview. 


Quantitative data would be hard obtain because of NSD regulations, seasonal variations, changing training schedules, violative members (ref Covid-19) et.c.

\section{Future suggestions}













Ut fra funnene ser vi at det kan være interresant å benytte seg av annen type nudging enn (signal). Facilitator burde bli undersøkt ettersom brukere understerker at de trenger noe mer konkret, ikke bare helseinformasjon. 
Det ville også være interresant å kombinere informasjonen med forslag, for å oppmuntre til både interaksjon med appen. Direkte kommunkasjon. gjøre dette - kommando. 

Annet type innhold, annen leveranse, annen mekanisme og bygge på andre biases og heuristics. 
Sammenligne ulike implementeringer for å finne ut hva som er det beste for denne konteksten. 

Future research should continue to test and evaluate various digital nudging implementations in real-world context, so that we can increase our knowledge of how to nudge through digital environments. As well as how we can make the best use of technology and human psychology to help and support good choices in relation to health. 

Digital nudging involves many components; mechanisms, what psychological effects should we utilize, delivery methods etc that work best for the given context. This study was the first attempt to evaluate a digital nudge intervention based on qualitative data, and has contributed a lot of new insights that cannot be revealed with quantitative studies.

It would be interesting to do similar user experience studies concerning other types of digital nudges, to get a insights from the users perspectives. Of course, it could strengthen the findings if the study was rearranged but with controlled user groups, including both inactive and active participants.   

Researchers should continue to do qualitative studies around users' experiences and perceptions of various digital nudge implementations. 

By applying the framing effect to digital nudging for physical activity we also saw other psychological effects... 

There is a need to explore other psychological effects that can be applied in digital nudging interventions, to make them more effective.

In future, researchers should use the design frameworks and guidelines, in order to implement digital nudge interventions, 

Even though this way of nudging (informational health nudges) primarily aims at the reflective mind, it would be interesting to see how such intervention would lead to over time.  For future work, it would also be interesting to look at different types and ways of implementation. In particular looking at those who talks to the automatic mind. 

%Hva hvis våre helsenudger hadde blitt broadcasted til hele norges befolkning? vi antar det ville blitt oppfatte i det negative leiet. 

 It would be interesting to include participants that are not currently engaging in physical activity but has the intention do to it in future, and then compare findings between the groups of active and not active users. Does inactive users perceive the digital nudging differently? How is their susceptibility? 

\subsection{Long-term effects}
It is known that studies regarding the long term effects is low for many HCI topics, including digital nudging. It would be extremely interesting to study this context in longer time interval, to see if this informational nudges had an impact over time. One of the participants actually expressed that "maybe over time, it would have a stronger effect on me". 