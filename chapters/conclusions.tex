\chapter{Conclusion} 
This chapter reflect on the main findings and research questions. It also states how the study have contributed in increased knowledge. Finally, suggestions are made on how further research should be based.

\section{Final Remarks}
If more time and resources were available for this research project, it could have been expanded on. For example by doing several iterations of the design phase, based on gathered user insights. The intervention should extend over a longer period of time to ensure sufficient grounds for mirroring real experiences and perceptions. By conducting pre and post interviews, one could have a better mapping of the participants, and selected candidates on the basis of personality traits. However, quantitative data would be hard obtain because of NSD regulations, seasonal variations, changing training schedules, violative members (ref Covid-19) etc.

Collaborating with a commercial player like Sats, made it possible to realize this project by actually performing a full scale intervention, including the design, implementation and evaluation of user's experience and perceived effects with this in a real context. Due to the scope of the thesis, this was only done in one iteration, but it would have been exciting to continue the study on several iterations to find out how such a digital nudge intervention should be tailored to different user groups within the target population, and even extend it to other target populations, such as inactive people. 

\section{Concluding thoughts}
This study aimed at investigating how a digital nudge intervention for physical activity promotion was received, perceived and experienced by users at a gym-center. By doing this, we gathered insights which contributes to the bigger question on how to nudge.

To the best of our knowledge, this study is one of the first attempts in investigating an actual implemented digital nudging intervention with a qualitative approach. It also strengthens HCI's role in this branch of the research field. Although the research on technology in the physical activity promotion domain has been extensive, this serves as the first study that defines/approaches intentional digital nudging within this context (could exist observational examples of digital nudging and PA). 

The technical solution worked as intended and it was possible to retrofit the push notifications to the existing app, there was no doubt about who was the sender. The sender Sats, being a commercial player, still was perceived as a credible source and even though the nudges evoked different feelings, there were few questions about the credibility of the contents.

Findings from the study could be important in order to map and establish more specific design guidelines for this context in the future. However, generalizations should not be made on the basis of the data, since a larger amount, and quantitative data is needed.

Although the project has encountered challenges, especially with regard to Covid-19 and the associated complications, changes and delays, it has met all objectives. By performing a qualitative assessment of the digital nudging intervention, we obtained important opinions that is not possible to identify in a quantitative study. The addressed research questions were successfully covered by data analysis and findings. For example, we have found that loss-framed nudges can cause negative effects for some users, which is highly undesirable. 

The contemplative and the reflective psychological effect can be found in many parts of the analysed data. Many stated that some of the information was new to them and that it may have started reflective and/or subconscious awareness accordingly. Other information serve as reminders, and had a particular impact when user could relate to it (as in tying it to real-life experiences). Younger participants, not surprisingly, seem to relate less to messages because of displacement. This coincide with the general perception that the predominant factor w.r.t. health risk is in any case age. 

Another important finding was that some experienced educational effect, which underpinned and expanded their reasoning for staying physical active. It is likely that this could lead to increased training frequency accordingly. 

%In a full rollout, i.e. more messages and more repetition, this effect could be confirmed and possibly strengthen. 

The digital nudging intervention was successful because the messages, and the opinions expressed, falls within the definition of nudging ; it should be experienced as a free choice, easy to opt out, transparent and should not enforcing. Participants appreciated that the nudges came without commitment. This was confirmed by the majority of the participants. 

A foundation for full scale nudge scheme is in place, and it is likely to have a quantitative effect in the long perspective. 

%The message framing was not an obvious property to the participants. It is probable that this would have become more recognize if the implementation lasted for longer and with more repetitions (i.e. rolling out same message again and/or rolling out positive/negative alternative of the same message). People were generally more positive to positive framing. Many stated that the messages were new information or served as reminders. The impact the messages made were to a large degree conditional as in relatable. Younger participants had a tendency to displace health risk information. 


\section{Contributions}
The study has helped elucidate digital nudging for physical activity context. 


\section{Recommendations for Future Research}
Future research should continue to explore intentional digital nudging interventions in real life contexts. It could however, be interesting to do a comparison between different implementations to understand which is the most suitable for this context. As the delivery method and health information seem promising based on main findings, it could be interesting to take these aspects further, while utilizes other nudging mechanisms (such as social comparison) and psychological effects (self-determination theory). 

Findings from the study also confirms and supports the need for tailoring in regards of content and timing to be more effective which we are already aware of. 

In addition, future research should target at examining digital nudging intervention with users over a longer period of time, as digital nudging often aim to alter decision making through awareness, and therefore it might have greater impact when users are exposed to the digital nudging for a longer time interval and more frequent.

Future digital nudging interventions should also follow the existing design frameworks and methods more strictly if specific user groups are investigated. It would be interesting to include participants that are not currently engaging in physical activity but has the intention do to it in future, and then compare findings between the groups of active and not active users. Does inactive users perceive the digital nudging differently?

%For future research it could be interesting to compare different implementations to find out what is best for this context. As the delivery method and health information seem promising for this context, it could be interesting to take these aspects further, while utilizes other nudging mechanisms (such as social comparison) and psychological effects (self-determination theory). 

%It is known that studies regarding the long term effects is low for many HCI topics, including digital nudging. It would be interesting to study this context in longer time interval, to see if this informational nudges had greater impact over time. 

%There is a need to explore other psychological effects that can be applied in digital nudging interventions, to make them more effective.
 



