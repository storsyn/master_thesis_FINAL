\chapter{Findings}
This chapter will present obtained data and findings from the thematic analysis, in form of key themes and concepts about users perceptions, that was derived from the thematic analysis, and also present some quantitative data. The themes reflect and represent patterns and prevalence across the data set. 

During the thematic analysis, several categories of perceptions and opinions related to the digital nudging interventions was identified. 

%Various ramifications were identified within the perception and experience of the implemented intervention: no effect (nudge was read but not carefully), minimal effect (nudge was read but not perceived as catchy), encounter effect (nudge was read and perceived as patronizing), incentive effect (nudge is read and provides affirmative effect).

By looking at data about participants perception of the digital nudge intervention, it was identified three main themes:

\section{Theme 1: The Competitive Feed}
%This theme embraces the opinions of participants who expressed that they read the nudges to a great extent, but failed to recite the content, i.e. it has not been memorized and the participants accordingly experienced minimal effect. 
%This theme hinges around what happens in the time between receiving and before actually reading the message portrayed by the digital nudges. More specifically , it reflects participants first glimpse and impression of the nudges and what determined if the nudge got attention or not. First step towards influencing choices and behaviors is through the way of communication / contacting / connect / approach : how to reach the target audience ? What attitudes do participants have about push notifications from previously  and how is this mode of delivery perceived in this context? 

\subsection{Attitude towards push notifications}
To understand how participants considered the received nudging we include some of the general attitudes towards the use of push notifications. Many participants expressed how they selectively considered which apps they allowed push notification on. (Note that part of the instructions to the participants were to allow push notifications from Sats specifically, or else it would undermine the purpose.) This was mainly justified by their experience of interference and noise from their phones during the day, which they wanted to reduce as much as possible, i.e. only the most important apps pass through the "allow push notification" filter. Further, general and advance attitudes were that push notifications are easily overlooked in the larger flow of information. 

Furthermore, it appears that people believe push notifications are an appropriate way of communicating content as long as it is perceived as useful. What the participants put in the term “useful” is of course varying; extraordinary messages, news, communication. Even though the addressed digital nudge intervention did not fall under this category based on their attitudes to push notifications, it nevertheless emerged that the vast majority were positive to using push notification to broadcast such information. One of the reasons for this was that general factual knowledge was conveyed and not cliche motivational talk which was associated with exercise/training apps. Also, several participants told that they were interested in physical activity and health in general and therefore considered the nudges as useful. 

Other attitudes towards the general use of push notifications was that if it comes too often, or with irrelevant content, it is perceived as irritating and disruptive. No one expressed that the frequency of nudges bothered or annoyed them, which is good considering that it does not overshadow / color / affect the perception of framing, content and delivery, which is the focus of this task.

Some associated push notifications with commercial content, advertising and not the least social media, and expressed prejudice towards such communications. Others expected it to be information that was urgent and vital and adjusted smartphone settings accordingly. Despite attitudes towards using push notifications to deliver information and regardless of the speculation as to what is urgent or not, they are largely welcomed in this digital nudge intervention, and there were several reasons for that. \textit{“Most push notifications are meant for the other party to profit from it, so I get	skeptical ... associate it with shopping pressure and advertising” (P9)}

\subsection{Easy catching attention}
Despite the overall attitudes and opinions about the use of push notifications in general, it appears that the nudges succeeded in catching participants attention (not every single nudge was given attention, but in the longer run), as everyone replied that they had noticed and read (to different degrees and extents) the push notifications from the Sats app in the given period of time. Participants expressed that whether they gave the nudge further attention after first glimpse, depended on their susceptibility. 

Timing and context were two factors that affected the extent to which the nudge received attention at the time it was broadcasted, i.e. if it was read there and then. Statements included that it was most often not payed attention to at school, at work and in busy situations. In addition, a minor part However, it may also be information (or other UI elements) presented to influence a choice in real-world context, e.g. associated activity meters that remind a user to move (what we will refer to as external digital nudging).  of the participants However, it may also be information (or other UI elements) presented to influence a choice in real-world context, e.g. associated activity meters that remind a user to move (what we will refer to as external digital nudging).  stated that getting push notifications from Sats was downprioritzed.\textit{ "It feels less important that rate that rate reminds me of something" (P8). }

\subsection{The Urgency Hierarchy}
Due to the special circumstances related to the Covid-19 pandemic and the lockdown of the society, a particular opportunity aroused with regards to investigating another category of digital nudging concerning health information. In specific, a SMS was broadcasted to all norwegian citizens on the first day of lockdown by Helsedirektoratet. It contained behavioural advices to limit the spread of diseases, e.g. coffing concerns, social distancing, recommendations on hand hygiene. All participants remembered getting the SMS nudge from Helsedirektoratet. Participants expressed this nudging as more direct and important, and that SMS was perceived as the right delivery method because it was so urgent. 
%trust/urgency source credibility/delivery channel hierarchy and 

\subsection{Readability and memorability}

Strong tendency that the nudges were read or partially read by the participants, however the timing for reading was situation dependent. That is, some actually read all while others had read only a few. The nudges were described as short, and easy to read. It required little from the participants in terms of comprehension. It was expressed by several that the nudges maintained participants freedom to read them at an appropriate time, and are therefore not perceived as intrusive. \textit{“It was such a short amount of information that I actually took the time to read it” (P8)}. Everyone claimed to have read the messages to some degree, and more or less all recognized two distinct attributes; 1) Sats, the training center was the sender, 2) There was some kind of health information involved. Further, some participants mentioned topics they remember better than others, i.e. specific conditions or health aspect that were addressed. A more exclusive group, namely one third (1/3) remembered and could recite details about of the different health topics presented. This was predominantly revealed when asking control questions. Statements were:  \textit{"I do not think they have had such a great impact, I have read the information, but did not think much more about it, the information was quickly forgotten" (P4)} and \textit{“When I receive it, I still have the choice to immediately read it or not, which is nice” (P1). }

\section{Theme 2: Food for Thought }
This theme embraces the consistently/common experienced effects of the digital nudging intervention. These findings are important because overall, users' perceptions were that it had some though stimulating effect. Even though these findings are common for the vast majority of participants, there were few strong opinions. 

\subsection{Limited but positive effect }
Participants themselves felt that the nudges had little or no effect on them, in terms of actually making the choice about physical activity after receiving the nudge. Only one participant described how a nudge made him actually go for a longer hike and scheduled a workout the following day. Others mentioned how the nudges planted a feeling that they ought to exercise, but could not answer whether it had actually lead to such outcome.

\textit{"Makes me think that maybe I should go to training today anyway" (P8).}

Still, it appeared that several participants implied that the thought of engaging in some sort of physical activity remained with them during the rest of the day. The extent to which this actually caused them to perform physical activity has not been possible to measure. For example, one participant said that in situations were the idea to workout was present, she would feel more committed to actually complete that workout after receiving a nudge. Further, some participants pointed out that they actually thought about the idea of having a training session (to what extent this was actually done is uncertain), i.e. the incentive was planted. Another participant said that a nudge made her reflect that she should spend an extra hour in training rather than on the couch. 

\subsection{Reminding effect }
The main takeaway concerning the participants perception and experience of the digital nudges, is that it initiated a thought process more than actions and behaviours. In other words, few participants experienced immediate effect, e.g. making a decision about engaging physical activity in the near future or planning a workout. However, the presented health information either aroused deliberate reflection (conscious) or got stuck in the back of their mind for the rest of the day (unconscious). So, even though participants did not decide to go directly to the training center or went for a run, it would stay with them as an consideration whether they should engage in some physical activity during the day. Another important effect of the nudges was that they reminded of the health effects of physical activity. The information about the health effects was already known (to varying degrees), and was also some of the participants justification for physical activity, yet not always in their mind. 

\subsection{Affirmative and infinitesimal motivation}
Another dominating opinion and experience with the digital nudges was that it had an affirmative effect, in the sense that participants got confirmation that their time and effort already spent in engaging in physical activity was worth it. It was a good reminder and support to make them want to continue what they were already doing. Also affirmative in the sense of the information was already known and agreed on, and was used as argumentation to stay physical active. Findings reflect that the digital nudges served as the infinitesimal motivation which had the potential to give the extra push to opt in for  physical activity. 

\textit{“They have served as a confirmation and an extra push that I should continue with what I do” (P15).}

\subsection{Supplementary information and Increased argumentation}
The majority mentions that most of information that was presented was already known, at least to a certain degree. Some of the themes (e.g. cancer) and level of details (e.g. strength workouts should be performed two or more times a week) turned out to be new for several participants. A few participants actually stated that they had expanded their argumentation to stay physically active based on the presented information. 
\textit{"I've added it to the reasoning to keep engaging in physical activity and training" (P9).}

\textbf{Opinions regarding health content} 
\textit{“It makes me think that I'm not only exercise for body and appearance but also for my health” (P3).}
%\textit{“It makes me think I should engage in physical activity based on these arguments as well.” (P3)}

\textbf{Source Credibility vs. trustworthy information }
Regarding credibility, the vast majority agreed that they could rely on the information that was conveyed. Some justified it because it was common knowledge that they already knew to some degree. Others pointed out that since it came from such a big player as Sats, they trusted the content. This was revealed through direct questions. 
“I find the information very trustworthy since it comes from Sats which I believe is the largest fitness center chain in Norway, it never struck me that I should be critical to the information” (P12)
“I do not doubt the information even if it comes from a commercial operator, I trust the information that is sent ” (P11)

\subsection{Susceptibility factors}
As the data indicates, the digital nudges are read because of its readability and effortlessness, but what happens next depends on…Susceptibility factors affects participants perception of the message, and thus further determines whether the information is considered and reflected on (?). Susceptibility refers to participants likelihood to absorb and reflect on the presented information. While and after reading the nudges, several factors impacted participants susceptibility, i.e. their further choice (unconscious and conscious) of acting, responding, or reflecting, on provided information. Our findings suggest three categorizations; 1) timing, 2) content-dependency (meaning that the perception of nudges is content-dependent) and 3) personal characteristics/psychological effects. 

\subsubsection{Timing}
Timing and context has already been mentioned as a factor to determine if the nudge receive attention or not. After read, the timing and context still has influence. Although a participant defied poor timing and context when receiving the nudge by reading the nudge, this assessment continues into how they respond further. That is, timing and context are two-way / two-layer barrier. \textit{“I think that could ever come back to what I haven't had time to reflect on, it's going to be a bit like the defense in everything about information, you register it, also disappear” (P8).}

\section{Theme 3: Contrasting Feelings}
As the name implies, there was identifies contrasting feelings among the participants, both regarding the perception of content and framing. 
\textbf{Conditional effect}
In addition, several participants described how the nudges could have been effective or more effective given this or that condition. This is mentioned both regarding the nudging in general, but also directed specific towards the message framing. Typical conditions that affected their perceived effect are: current training routine, prior physical activities frequency, work situation and free time, prioritizations, timing, lack of motivation. Some of the statements were: 
"If I had not trained normally and would be positively influenced to exercise, I think that is the kind of information I would like to be notified of ” (P22)
"The periods when I work a lot of overtime it is a little more demanding to get into training, if I had received one of the alerts in such a situation it might have influenced me to go  training anyway" (P19)
"If I had gotten them at the right time, e.g. the days I am struggling to get to workouts, they might have had a bigger impact on me" (P18)
“It depends on whether I'm in a good or bad period of training. If I had been in a bad period then maybe the other (loss-framed) would have pushed me a little more." (P2)
These are augmentations are only hypothetical, meaning that it is not the current experience with the nudges. Makes it weaker than the answers where they do not express conditions to their answer. This is a group that shows how tailoring of content and timing is important. Another conditional effect, was that some of the users that expressed positive statements regarding the digital nudge intervention, also expressed that they thought it could have an even greater effect if they were exposed to it over time.

\subsection{Content}
\subsubsection{Nearby vs. future}
There was a strong tendency among the younger participants (under the age of 30), where several stated distance to many of the presented health effects. They reflected that the information was in fact important, but they displaced it because it felt too far away from their stage in life and distant to make that the argumentation to engaging in physical activity. Typical topics that was displaced by youngers was; cardiovascular diseases, muscle and skeleton and cancer, life expectancy. 
\textit{"I feel that cardiovascular disease and cancer are a bit far away from my concerns" (P11).}
\textit{"The notifications appeal more to the long-term version of myself, and not the here and now version" (P12).}
\textit{"When you are young and healthy, cardiovascular disease feels too far away for making it an argument for engaging in physical activity” (P14).} 
Older participants did not seem to distinguish some topics as less relevant than others. Further, regardless of age, health effects that were easy to relate to and felt close, were perceived as more effective. (i.e. leading to greater potential effect). Such health effects was how physical activity affects the participants "here and now": the brain, stress management, endorphins and dopamine and mental health. 

\subsubsection{Novel vs. familiar}
Older participants (over the age of 30) express that nudges presenting new information was more interesting as they could learn something from it. \textit{"had I been able to learn something from it, it would have been interesting" (P5).}
While other participants thought it was nice and most effective to be reminded of the health effects they had already imposed as an argument for participating in physical activity (i.e. familiar information), as this was not something they were constantly thinking about. 

\subsubsection{Training app expectations}
Another perception regarding content was that participants experienced that the nudges did not encourage any concrete or specific actions, choices or behaviours. The fact that participants had to choose how to use this information, act or respond to it, was experienced as both good and bad. Some participants experienced this as commitmentless, and wanted more concrete information, such as challenges to join, exercises, type of training, etc. Whilst others appreciated the fact that they had the ability to choose how to act on the information. Participants desired freedom of choice and ownership of the choice for engaging in physical activity. 

\subsubsection{Level of appreciation}
A few participants brought up the fact that they had not requested this information to be delivered (expect from signing up  to participate in this research), and claimed that they could take the responsibility of being aware of such information themselves. In other words, they didn't want it since they hadn't asked for it and experienced it as unnecessary and redundant. On the other hand, other participants expressed how it was nice to be reminded, and were happy that the information was provided for them without having to actively look it up. 

\subsection{Framing}
\subsubsection{Strong negative vs. weak positive}
%As we have already mentioned, some participants put limitations on potential effectiveness of the digital nudging intervention. This also involved the effect of different message framings (gain and loss). “It depends on whether I'm in a good or bad period of training. If I had been in a bad period then maybe the other (loss-framed) would have pushed me a little more." (P2)
When participants were asked open questions about what they noticed about the digital nudges they had received, the framing was not mentioned as a property. Due to only receiving 8 nudges, the perceptions, opinions, attitudes and experiences regarding this were vague. Therefore, as an alternative to the intended broadcast schedule, the digital nudges were presented to them in the chat, during the interview. They identified nudges they thought were interesting and received both the gain and loss version of the same message. Without pinpointing the difference in framing, they were asked how they perceived the two versions, and which one that would have greatest effect on them.

%\begin{comment}
\begin{table}[ht]
\begin{center}
\begin{tabular}{|c|c|c|}
\hline
\textbf{Gain} & \textbf{Loss} & \textbf{Indifferent} \\ \hline
15 & 3 & 4 \\ \hline
\end{tabular}
\caption{\label{tab:table-name} Participants categorized by which message framing they experienced as most effective.}
\end{center}
\end{table}
%\end{comment}

\bigbreak
\textbf{Gain-framed}
\bigbreak
A majority of participants experienced gain framed as most effective, because it was perceived as positive, uplifting, rewarding and had a motivational effect. However, their opinions were portrayed in a neutral manner (low voices and engagement), meaning that when they expressed it could have an motivating effect, it was still small. It was also pointed out that it was easier to relate to the positive health effects, especially those regarding nearby topics such as mental health and endorphins and dopamine. 

\textit{“It tells me what physical activity actually gives me, uplifting in style, feels like it rewards me for exercising.” (P2)}

 No participants expressed negative opinions about gain-framed nudges.
\bigbreak
\textbf{Loss-framed}
\bigbreak
A few participants expressed that they considered loss-framed nudges to be more effective due to its serious tone. The perceived negative connotations made a deeper impact on them. 

\textit{“It depends on whether I'm in a good or bad period of training. If I had been in a bad period then maybe the other (loss-framed) would have pushed me a little more." (P2)}
\bigbreak
\textbf{Indifferent}
\bigbreak
Further, a few participants answered that they did not see the gain-framed or loss-framed as more effective than  the other, and argued that other factors, such as content and health topics was more important for their perceived effect. 

%It seemed that people had more opinions about loss-framed than gain-framed. When asked what they thought about the different versions, people often had a lot to say about loss but little to say about gain, yet they claimed that gain had the most effect on them (when it comes to motivating for physical activity).
%The above mentioned findings emerged as we specifically asked them questions regarding their perception on gain and loss (without mentioning or explaining the gain loss term), in order to collect data for the research question (RQ1.a). 

These findings was identified by directly presenting participants with the different framings as it was relevant to the research question (RQ2). However, it turned out that the message framing itself was in fact not noticed by most of the participants themselves, except those who had very negative opinions related to the loss-framed nudges. What turned out was that participants were more opinionated about the varying health effect topics that was presented than the framing.

\subsubsection{Counter effects}
As mentioned, the dominant overall perceived effect is that the nudges act affirmatively on current training and that they act as a push to continue, which shows desirable effects with the implementation. However, when looking more deeper into the impact of message framing in particular, the complete opposite emerges for a few participants. 

Firstly, several participants could identify the negative connotation in the loss-framed nudges when presented side by side. Some of them expressed that presenting the negative effects with not engaging in physical activity was not effective on them because it could lead to bad conscience, scare, warning and felt like they were given a penalty if not staying physically active. \textit{“This one is a little more threatening, warning, tells about the penalty of not exercising.” (P2)}. It was revealed that those who were negative to loss-framed were more committed and clear on what they thought, than those who were positive towards gain-framed. Some did not express any strong opinions, but could still identify the negative framing. And among them, a smaller group of participants expressing rather strong opinions.

On the other hand, negative opinions were expressed about loss-framed nudges in particular. This was the case for a smaller group of participants, but at the same time they expressed stronger opinions (by using more powerful language, several words to describe their opinion, higher voices and seemed to be more engaged than any other participants), and are therefore an important finding. One participants that had strong reaction on it, said that he experienced the loss-framed nudge as patronizing and commanding, which made him demotivated. One participant had extremely negative opinion. The user express that he perceived it as patronizing, a counter effect what we actually want to achieve
\textit{“It feels like I'm being forced to train and yelled at. Such messages does not motivate me to engage in physical activity or workout. To be honest, I am demotivated. The way in which the message is presented makes me tired. It feels like an order and that the push notification knows best.” (P16)}

\begin{comment}
\begin{table}[ht]
\centering
\begin{tabular}{lllll}
\cline{2-4}
\multicolumn{1}{l|}{}                               & \multicolumn{1}{l|}{\textbf{Gain}} & \multicolumn{1}{l|}{\textbf{Loss}} & \multicolumn{1}{l|}{\textbf{None}} &  \\ \cline{1-4}
\multicolumn{1}{|l|}{\textbf{Extrinsic motivation}} & \multicolumn{1}{c|}{10}            & \multicolumn{1}{c|}{2} & \multicolumn{1}{c|}{1} &  \\ \cline{1-4}\multicolumn{1}{|l|}{\textbf{Intrinsic motivation}} & \multicolumn{1}{c|}{5} & \multicolumn{1}{c|}{1} & \multicolumn{1}{c|}{3} &  \\ \cline{1-4}
&&&& 
\end{tabular}
\caption{\label: Shows how many participants preferred the different wording of the message.}
\end{table}
\end{comment}



















