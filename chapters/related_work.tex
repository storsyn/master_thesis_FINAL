\chapter{Related Work}
Testing av referanser \cite{acquisti_nudges_2017}, \cite{al_stairs_nodate} \cite{suri_stairs_2014}, \cite{fogg_persuasive_2003}, and \cite{hamper_behavior_2016}. +  \cite{karl_chapter_nodate}
%This master thesis had to adapt aims, objectives and research question after the pandemic hit Norway and caused problems with implementation of the study as it was originally designed. This chapter is strongly affected by this, and does therefore include some theory on topics (e.g. message framing) that are not elaborated on or discussed any further. 

This chapter aim to give an overview of supporting theory, relevant articles and topics that this study is founded on. As both persuasive technology and digital nudging are extensive research fields, the literature is limited to address mobile applications and physical activity promotion. This means that studies on nudging for wearables such as activity trackers have not been considered.

\section{Health}
Health is a prominent application domain and area of interest within the human-computer interaction field, both when it comes to health informatics, healthcare technologies and personal informatics for health and wellbeing (Reference: Technological Approaches to Promoting Physical Activity, HCI and mobile health interventions, HCI for health and wellbeing: Challenges and opportunities). There is a large potential for motivation, encouragement and the ability to change people’s health-related behavior through user interface design as people are becoming rapidly more dependent on technological and digital interfaces. Over the last decades, the development of health technology that promotes or helps people to physical activity has evolved from completely primitive step counters to wearable fitness trackers, often with a corresponding persuasive mobile application to visualize activity data (reference: ?). Now, the training apps evolves towards full integration with both activity tracker devices such as heart rate device, smart watches and gym membership. 

Evolution of technology for different health promotion started of with elementary step counters, large, clumsy artifacts that could be perceived as disruptive in everyday activities, to become step count as the most primitive functions that is embedded in almost all ubiquitous and wearable devices nowadays. Next were the more advanced fitness trackers, smart and pulse watches, and most recently development of persuasive mobile applications and versatile health wearables and integrated gym app.

Visualizing, tracking, monitoring and logging activity data motivates people to train more (reference). More recent design principles such as gamification, feedback, rewards, challenges/competitions and social influence has been embedded into the fitness devices. (Reference: The Potential Role of Digital Nudging in the Digital Transformation of the Healthcare Industry).
 
Around the peak in persuasive research (~2010), several articles aimed to give an overview of the technological approaches in different health related contexts. One particular article that mentions current and potential contributions in promoting physical activity at the time, was conducted by Maitland and Siek (Reference: Technological Approaches to Promoting Physical Activity, 2009). Monitoring capabilities and activity inference techniques. Effective user experience. Applications that provide monitoring of physical activity often combine it with other different techniques to encourage users to be more physical active, for instance through goal setting, feedback and social influence. The article states that physical activity promotion and support systems have the power to lower the barriers for physical training, but cannot solve this challenge alone. Health education is another way of solving this problem, at least in certain parts of the world where demographic and technological prerequisites are fundamentally different.

\subsection{Barriers to Physical Activity} 
In order to develop and define guidelines for nudging mechanisms regarding physical activity promotion //  and understand how we can design effective nudges in digital contexts to promote physical activity, one need to visit the barriers which prevent or counteract physical activity. Such barriers are strongly dependent on several demographic parameters; age, gender, geographical location, nationality, cultural background, educational level, literacy, social and economic class and other (Reference:Perceived barriers to physical activity across Norwegian adult age groups, gender and stages of change - ok å bruke den kilden? eller må man egentlig ha kilde på det?) . Due to this, physical activity barriers are often researched within a limited scope, and with very specific user segments, such as diabetics, elderly, pregnant, college students, etc. However, it is reasonable to evaluate lack of motivation, information, time (hva mener jeg her?), and injuries or disabilities, as “general” barriers for accomplishing physical activity independent of age and geographical location (Reference: physical barriers). Both lack of motivation and information (are examples where technology to motivate and educate is utilized) are manageable to overcome, because one can educate and/or motivate people through nudging. People suffering from serious diseases will probably not, to the same degree, benefit from nudging because motivation is hardly a factor. However, people that suffers from motoric diseases, chronic conditions or similar disorders that require self-management/care tools, could for sure be assisted by nudging (Rouyard et al. 2018). More on this is following section ?

\section{Persuasive Technology}
As mentioned in the introduction, a concept that is popular and accentuate nowadays, in all fields of informatics, design and psychology, is persuasive technology. Within this collective term there exists a numerous applications, systems and softwares, which intentions are to persuade users to act on something. That something could for instance be to stop smoking, drink more water or avoid snacking. Persuasive technology aims to change people’s behaviour, attitude or habits towards the most beneficial for the individual, through interactive systems (Reference: Orji and Moffatt 2018 - Persuasive technology for health and wellness: State-of-the-art and emerging trends). Based on a variety of strategies and principles, the interface is designed to assist the user to gain motivation or interest for the sought behaviour change in a given context. This research field was introduced around 1990, by one of the discipline pioneers, B.J Fogg, a behavioral psychologist. His contribution became the very foundation for further research and proceedings. He stated that the power of designing for behavior change, should only be used towards enhancing society and people's life.

People are situated amongst and exposed to persuasive technology everywhere in the society, without necessarily being aware of it. The most typical application domains are education, traffic, sales, religion, sustainability, insurance, military and public health. The latter, persuasive technology influencing health and wellness, has received vast attention and lately expanded on its targets (Orji and Moffatt 2018) (Persuasive technology for health and wellness: State-of-the-art and emerging trends). For what pertains to health and wellness, two main directions have evolved; 1. disease management and 2. health promotion / prevention. Disease management hinges around enhanced self-monitoring and management for individuals that has already contracted an illness, injury or disease. It could for instance be self care technology for diabetics (Rouyard et al. 2018) or self medication to cure HIV (Schnall et al. 2015). As for the other direction, i.e. persuasive technology for health promotion or prevention, it addressed the proactive and inhibitive aspects of physical and mental health. It could prevent diseases through promotion of physical activity or healthy dieting. It can also be used to encourage self diagnostics, e.g. self monitoring of mole cancer, breast cancer or similar, thus detecting symptoms at earliest  possible stage. In this thesis, I will mainly focus on health promotion and prevention, specifically physical activity promotion. 

It should be noted that the persuasive technology and design sphere in the health domain embraces several categories. There exist numerous different devices, ways of implementations, user interfaces designs, e.g. step counter, fitness trackers and other wearables, exergames etc (Reference: Technological Approaches to Promoting Physical Activity, 2009). However, this review will be limited to concern the physical activity domain and PT within mobile applications, as these are the more relevant to this master thesis. 

\subsubsection{Persuasive design for Physical Activity}
According to authors X, motivation is a powerful factor when it comes to physical activity promotion. One study that underlies this (Behavior Change Support for Physical Activity Promotion: A Theoretical View on Mobile Health and Fitness Applications) views mobile applications for behaviour change. From Fogg's behaviour model, we already know that motivation is one of the components that lead to behaviour. However, motivation is a complex matter as it could be a mix of both intrinsic (internal rewards; joy) and extrinsic (external rewards; appearance) factors. (find reference ??)

A quite recent review of persuasive technology for health purposes, identified 85 researches within 16 years of time (Reference: Persuasive technology for health and wellness: State-of-the-art and emerging trends, 2018). It stated that physical activity was the most researched health domain for persuasive technologies. However, only 15 percent followed an adequate qualitative research approach, which indicates that there are relatively few studies on user experience and perception when it comes to persuasive technology for health domain, including physical activity promotion. The effectiveness of different persuasive technologies was in particular investigated... 
Although different theories from behavioural science and motivational strategies were utilized, it appeared that persuasive messages, reminders and alerts (which all are relatable to digital nudge interfaces) were not identified in the research concerning physical activity. This further implies that a gap exist in the research.

Another study only reviewed persuasive technology for physical activity for mobile applications (Reference: Persuasive Technology in Mobile Applications Promoting Physical Activity: a Systematic Review, 2016). Articles regarding persuasive technology for physical activity was evaluated by using the PSD framework. Conclusion was that different design principles from the PSD framework seem to be dependent on each other. 

\subsection{Design Principles}
During the years of research on persuasion in technology (three last decades?) , different guidelines, principles and frameworks have developed. The study by Oinas-Kukkonen and Harjumaa (Reference: PSD: Key Issues, Process Model, and System Features), is recognized and widely referred to in the field persuasive technology. It provided a framework for designing and evaluating persuasive systems. Four dimensions of support were identified; primary task, social support, dialog support and system credibility. Within these dimensions, 28 underlying design principles were announced. Some of these design principles will be reviewed. 

\subsubsection{Tailoring}
A prominent design principle within persuasion is tailoring, which have been widely investigated (Reference: Persuasive system design : State of the art and future directions). It belongs under the dialog support dimension, and refers to a systems ability to achieve higher level of persuasiveness when content and services are tailored to personal factors such as needs and characteristics, or contextual factors such as usage. Typically this data is information about country, language, location, etc. Using such data to personalize the service or application, are design principles that almost every commercial company take advantage of today. Online streaming services, webshops, grocery shops, travel agents etc all do it to different degrees. Comercial players are using these elements as effectively as they possibly can within the boundaries of technology and legislation. Thus health promotion initiatives involving nudging may have benefits of using same. No different from other applications. ln the health sector there is a fine line between what can perceived as commercial interest and libertarian paternalism. What is the motivation to implement it in any case? To get more engaged users, less hassle and time consuming, increase effectiveness.The effectiveness of the technology is the true measure of its value and relevance, 

\subsection{System Credibility}
Other design principles from the PSD framework hinges around the dimension of system credibility. It is known that PT with features of system credibility will achieve higher level persuasiveness. (Does this hold for digital nudging as well?)  

Design principles belong to the Dialog support dimensions (such as reminders) could in fact remind of digital nudging. ...: minner om digitale nudger? Nudger kan bli sett på som en del av dialog support feature. Eller kan sammenlignes med de. 

\subsection{Communication}
Whenever technology is used as a medium for persuasion, it becomes natural to visit the aspect of intentional communication (Reference: Persuasive System Design: State of the Art and Future Directions). Fields of computer-mediated communication, rhetoric and psychology are strongly related to the HCI community, as they overlap and complement each other in the "How to design" questions. 

%Is the aim always to persuade? Is there different ways to persuade? Short-term and long-term? Is it about convincing the small and direct choices (here and now), such as whether you should go  training today or not. Or the long-term choices, to maintain a healthy and active lifestyle. 

Although one may say that research on persuasive technologies and communication had its heyday around 2009, it has remained an important topic and  branches into new directions. 

In fact, how we communicate is a powerful factor when it comes to technologies for behaviour change and nudging interventions (both offline and digital), since it often boils down to how the possible choices and directions are presented. One of the most mentioned communication strategies within this research area is framing, which will be elaborated  later on.

\section{Digital Nudging}
In recent years, a new strategy for design with intention has emerged in the siding of persuasive design; digital nudging. The concept refers to influencing/altering/guiding/fostering people's behavior and choices through purposeful application of user-interface (UI) design elements in digital environments. Such elements could be graphics, wordings, contents, etc. and thus often represent simple and cost effective modifications in the user interface (Reference: Making digital nudge applicable). 

This design strategy origins from the term nudging within behavioural economics, and was coined in 2008 by Thaler and Sunstein (Reference: Nudge book). Back to the roots, nudging refers to influencing behaviours in desired ways through a deliberate presentation of choices, without using elements of coercion or changing economic incentives. In other words, preserving users freedom to choose is an important aspect of a nudge (i.e. easy to resist). Since then, nudging has frequently been applied for different purposes and domains, such as government (e.g. in the fight against the coronavirus) and health (e.g. hand hygiene at work). In the matter of fact, a clear/distinct/ application of nudging has been observed the past months, in the fight against the coronavirus. This proves the timely relevance of the concept. 

Gradually, different nudging techniques were passed on to technological and digital interfaces. Based on reviewed literature it seems that nudging through technological devices was first discussed around 2013 (Reference: Nudging through technology). At this time, the concept was called technology-nudging, which embraced systems and technologies that aimed at helping users make better choices at individual and societal level (e.g. enhancing health), by providing necessary domain knowledge. However, it was not before 2016, that the term digital nudging was defined by Weinman, Schneider and Brocke, now specified to the use of user interface design (reference: Digital Nudging, Weinman, Shcinerd, brocke). Further, in 2017, digital nudging was introduced for HCI and IS domains (Reference: Digital nudging altering user behavior in digital environments). It is therefore a quite novel, yet prosperous, research area, touching upon several fields of expertise.

Several well-known concepts and cases can be considered as digital nudging (such as prompts, reminders, push notifications, etc). In fact, digital nudging often appears in practise, without consciously applying digital nudging strategies during design and implementation (Reference: Digital nudging altering user behavior in digital environments). You may have noticed them without being aware, small subtle changes in the choice environment on your phone or at the web. Supposedly they are everywhere around us. 

As digital nudging is partly overlapping and share properties with principles/purposes of persuasive and behavioural design, it can be hard to distinguish between them (whether this is desired at all?). It appears that no successful attempt at delineating the terms by describing the real relationship between the two, has yet been made. The main difference between them seems to be that digital nudging is supposed to build upon psychological effects such as biases and heuristics (Reference: Making Digital Nudging Applicable: The Digital Nudge Design Method), (which is not the restrictions for persuasive design principles even though many of them does). It is also suggested that digital nudging actually performs a subspace of the design aspect if persuasive technology (Reference: Digital Nudging: Guiding Online User Choices through Interface Design, 2018).In addition, incorporated in the definition of digital nudging, is that the users freedom of choice must be honored. 

%One example of conventional nudging for physical activity promotion is; steps were painted at the staircase so that people would choose the stairs instead of the elevator (reference). 
%findings and insights from similar research fields such as behavioral economics and social psychology, and on similar concepts, such as persuasive technology, should be addressed.  

Considering the novelty of digital nudging...

Despite the fact that digital nudging has been predicted as a fruitful research area for HCI (Reference: Digital Nudging – Guiding Judgment and Decision-Making in Digital Choice Environments), the current literature (referring to digital nudging in its true state) is mainly limited to the IS field. The work by Caraban et al. appeared as a necessary step forward by highlighting different approaches for nudging through technology, i.e. contributing as a framework. However, it also states the need for more research in the direction of establishing a complete and common understanding of the design space. To the best of our knowledge there is a lack of studies that actually carry's out the design, implementation and evaluation  of digital nudging interventions (by its true forms), with the aim of understanding users' perception and experience of digital nudging.

%To the best of our knowledge, the research on digital nudging for physical activity promotion is also a thin literature for now. 

\subsection{Complexity}
Since the research of digital nudging is a relatively new art, the exploration of nudging is still developing. Whilst some aspects are covered, there are still things that we yearn to find, and the complexity of digital nudging emphasizes the need for extensive research. Some of the design aspects of digital nudging will now be reviewed...

\subsection{Digital environments/technology channel}
Digital nudging unfolds within multiple digital environments; mobile applications, websites, wearables and other user interfaces (Reference: Digital Nudging – Guiding Judgment and Decision-Making in Digital Choice Environments). Hence, the concept of digital nudging is broad and includes several ways of implementation. This literature review will focus on digital nudging within mobile applications due to the scope of this research project. 

Even if digital nudging via mobile application may seem like a narrow variation of the art, it can still be implemented in different ways. 

It could appear as an internal quality integrated in the mobile app interface. Most often such nudges will aim to guide / help / influence digital (in-app) choices, such as fostering users to make appropriate choices pertained to the app itself and/or the services it facilitates. 

One may regard this as an inherent/internal quality of an app but it is predominantly related to digital choices. E.g. defaults options when setting up a new phone, opting users to make good choices w.r.t privacy and security, etc.

Digital nudges can also be delivered from an external source or database through different communication protocols/technology channel/delivery method (delivered outside the app, as an additional feature, but still belong to it). It can be described as information/content with the purpose of stimulating good choices that is associated with physical world and behaviours, e.g. push notification reminder to drink water (what we will refer to as external digital nudging). The content may consist of well known UI elements like text and graphics. 

When regarding the categorization of integrated and external digital nudging, it should be mentioned that this is not defined in previous literature, but it is our suggestion and contribution to the understanding of digital nudging, in order to distinguish the different variants that emanate from digital nudging. This thesis will focus on external digital nudging. 

\bigbreak
\bigbreak
\bigbreak

Another aspect of (external) digital nudges is the technology channel they are mediated through. Websites, mobile applications and wearables are examples of a technology channels (aka digital environments or digital context). Further, nudging mechanisms could be implemented in different ways within the chosen technology channel. For example, in mobile applications the nudging mechanisms could be integrated into the user interface itself (e.g. defaults, online forms) or delivered to the user as external prompts (e.g. email, vibrations, SMS, in-app notifications, push notification, sound, light notification, digital badges, etc) (Reference: Functional Digital Nudges: Identifying Optimal Timing for Effective Behavior Change). Authors of (Reference: Functional Digital Nudges: Identifying Optimal Timing for Effective Behavior Change) discuss that more investigations are needed to establish which delivery method should be utilized for specific target behaviors. 
In order to design effective digital nudges, an assessment of available technology channels and delivery methods should be conducted (Reference: Making Digital Nudging Applicable: The Digital Nudge Design Method). It is important that the chosen technology channel and method are familiar to its audience. It might be the case that different methods are experienced differently in terms of credibility, intrusiveness and usefulness to users. Looking at other qualitative studies on the user's experience of the various technology channels and delivery methods could be useful. 

For physical activity promotion, the use of text messages is widely researched (however not as nudging) (reference: Efficacy of text messaging-based interventions for health promotion: A meta-analysis). 

To the best of our knowledge, there are no studies scrutinizing push notification as delivery method for digital nudges even though several studies mentions it as a commonly used. This may be due to the novelty of research, and because many of the examples of nudging are unintentional

(Other considerations is as mentioned timing and frequency, however, this was not the focus for this thesis.) 


However, relevant studies state The use of push notifications should be carefully considered, to avoid being perceived as disturbing or troublesome to users (Reference: Push Notification Mechanisms for Pervasive Smartphone Applications). The article mentions how the use of push notifications has increased in marketing, to push advertising and offers to users. Other typical use of push notifications is to remind the user to interact with the (sender) app, i.e. to increase user engagement, which is a challenge regarding health promoting apps (Reference: To Prompt or Not to Prompt? A Microrandomized Trial of Time-Varying Push Notifications to Increase Proximal Engagement With a Mobile Health App).

\subsection{The many facets of digital nudging}
Some researches have aimed at identifying different types of nudges, and for which situations they are most effective. However, it seems that there is a lack of consensus when it comes to the different types of nudging, due to researchers from different communities defining new and different types in a rolling manner. Furthermore, the labeling of the nudge categories does not match even though they talk about the same concepts. 

Sometimes nudging can be difficult to identify as it overlaps with other design strategies and principles. To make it even more complicated, nudging is more like an umbrella term that comes wiht a numeberr of different appearances and facets.Through the systematic review on technology mediated and digital nudging by Caraban et al. (Reference: 23 ways), 23 distinct mechanisms were identified. The mechanisms where classified in six different categories: facilitate, confront, deceive, social influence, fear, reinforce. 

\begin{itemize}
\item The facilitate nudging mechanism, reduces people's physical or cognitive effort, hence facilitating for decision making. It is based on the status-quo bias which refers to that humans tend to go with the option with least resistance.
\item Confront nudges introduce doubt when people are about to make a decision. It exploits the regret aversion bias, which means that people are extra careful in a situations where risk is perceived. By breaking mindless behavior and make suggest using the reflective mind. 
\item A deceive nudging mechanism adds inferior choices so that a particular choice is promoted harder than the others. 
\item The social influence nudge takes advantage of what people believe is expected from them, such as adapt to or accept certain norms or social rules. 
\item Nudges in the fear category, uses the feeling of fear or loss to make... 
\item Reinforce: repeat actions to learn reflective thinking. 
\end{itemize}

Further, Caraban et. al (2019) makes an attempt to define in which situations the different nudging mechanisms should be used. Through Fogg's behavior model they are able to categorize all the 23 nudging mechanisms in one of the three forms: facilitator, signals and sparks.

\begin{itemize}
\item Facilitator: (ability strategies), makes an task or a choice easier for users. Reducing distance and effort. 
\item Spark: (motivational strategies), says to be optimal in situations where the user lack motivation 
\item Signal: (trigger strategies), optimal in situations where the is a divergence between a users intention and actions, but where motivation and ability is often present. 
\end{itemize}

Despite the research already covered on the topic, a complete overview of the different types and implementations of digital nudging is still missing. Other researchers have categorized nudging in a different approach, sorting them within other types of labels. For example, Jespersen and Hansen, defined four different classes where they categorised the nudges in (Reference: Nudge and the Manipulation of Choice. A Framework for the Responsible Use of Nudge Approach to Behaviour Change in Public Policy). Two questions determines their four classes: if the nudge is aimed at the reflective or the automatic mind, and if the nudge is visible for the user or not (also called transparency). This creates four intentions of nudges; automatic-transparent, reflective-transparent, automatic-non-transparent and reflective-non-transparent.  

\begin{itemize}
\item Automatic-transparent: nudges that intend to influence behavior (e.g., changing the default option)
\item Reflective-transparent: nudges that intent to prompt reflective choice (e.g. the“look right” painted on the streets of London)
\item Reflective-non-transparent: nudges that intent to manipulate choice (e.g., adding irrelevant alternatives to the set of choices with the goal of increasing the perceived value of certain choices)
\item Automatic-non-transparent: nudges that intent to manipulate behavior (e.g., rearranging the cafeteria to emphasize healthy items)
\end{itemize}

Since different categories and labels are used on seemingly similar nudging mechanisms, it seems that there is a need of a structured overview of the digital nudging hierarchy. 

\subsection{Digital nudges as signals}
In order for nudges, belonging to the category of signals, to result in the achieved targeted  behavior or decision, some variables should especially be considered. Those variables are timing, frequency and tailoring (Reference: 23 ways). A signal should appear in the right moment in order to be successful, however, what characterizes the right moment will depend on the situation and purpose of the nudge. Caraban et al. also claims that frequency may affect the effectiveness of the signal. In addition, another study showed that tailored signals / prompts / reminders / nudges with higher frequency are more effective than generic prompts with rarer frequency (Reference: Periodic Prompts and Reminders in Health Promotion and Health Behavior Interventions: Systematic Review). 

Within tailoring, it also appears different variables (for instance content and framing), which leads us on to the common design principle of persuasive design. Which we are describing later on.  
Could be similar to prompts, push notifications, reminders, etc. 

\section{Psychology behind }
What makes persuasive technology so powerful? One of the answers in order to manage and understand this, is the fact that it builds upon theories from both psychology, cognitive science and behavioral economics. Together with developed design features and principles within persuasive design, it underlie how the technology is experienced and perceived as convincing. Firstly, I will describe some of the essential details from psychology that aims to explain how people think, and then some design principles and motivational strategies, in the following paragraphs. 

\subsection{The Dual Process Theory}
Human decision making is compounded process. The Dual Process theories of decision-making is an important contribution in order to understand the human behavior (Caraban et al. 2019)(reference:  Thinking Fast and Slow). Humans have two modes of thinking: the automatic and the reflective. The reflective system is based on rational processes and shaped by consciousness. This means that it is slow, effortful and goal-oriented. According to reference, most of HCI research (94 percent) focuses on this way of thinking when exploiting the effects of nudging. Nevertheless, it is the automatic system which is used in most cases during a day (95 percent). It is responsible for repeated and skilled actions, and works therefore quick, needs minimal effort, and is based on instinctive, emotional.  unconsciously).Including research on the automatic system in the context of nudging, is referred to as an untapped areas of opportunity.

Humans are irrational and tend to take choices accordingly. Sometimes we need
a gently push in a certain direction, to take more sustainable, healthy and expedient choices for ourselves and the society. Inactivity and public health is a worldwide problem  and  is  often  used  as  the  purpose  to  study  the  effects  of  nudges.   Most  studies regarding physical activity promotion exploits the reflective mind (reference), whilst the idea of being able to promote motivation for physical activity to be handled by the automatic system could be revolutionary. In fact, the literature rise questions if technologies can rely less on users will and capacity to engage, meaning exploiting the automatic system (caraban det al?) 

\subsection{The framing effect}
Human forecast is proven many times to be biased, and our decisions is inveigled by cognitive biases and flawlessness  \cite{thaler_nudge-_2009}(Reference: Nudge Book). Nudging can help the individual to make optimal choices by detaching from systematic cognitive biases (Reference: Nudge Book). Our decision-making is therefore influenced by how the choice is presented, i.e. its framing. Framing could for example be the use of different situations, wordings or consequences, to convey a message so that it is presented in a way that makes the choice more attractive for the user. 

One way of utilizing the framing effect is through message framing/wording of a message. Message framing, meaning how the message is presented, either with the positive consequences (gains) of a certain action or the negative consequences (losses) with the absent of that action \cite{rothman_shaping_1997} (Rothman and Salovey, 1997). To understand the effect of framing fully, it is necessary to visit other theories within psychology that strongly relates to framing. Prospect theory makes statements about gain-frames and loss-frames as to which is better to use faced with a quantity of risk, e.g. often denoted as small risk or large risk (smaller or bigger evil) (Kahneman and Tversky, 1979). In context of physical activity promotion it can be translated to the large risk being the risk of contracting a health condition with the ultimate consequence of (early) death. The small risk can be interpreted as the time wasted in the gym should this information not be correct. Under such circumstances theory states that individuals a willing to accept the small risk (e.g. waste time in training) to avoid the big risk (early death) and that positive framing is favourable. 
%This is also reflected through loss aversion bias, which claims that there are stronger reactions and emotions associated with a loss than a gain (reference) .

In the context of health related behaviours, message framing has in particular showed to be effective. For example, nudging people to do self examinations for skin and breast cancer (detection behaviour) have proven to be best when the framing of the message enlights the increased risk in the absence of it, and not the reduced risk if they actually do it, i.e. loss-framed (Reference: Nudge book). On the other side, several studies conclude that in contexts of preventative health behaviors (such as skin cancer screening, smoke cessation, physical activity), gain-framed messages are measured to be most effective compared to loss-framed \cite{gallagher_health_2012} (Reference: Health Message Framing Effects on Attitudes, Intentions, and Behavior: A Meta-analytic Review)

A systematic literature review on existing nudging research (primarily offline context) reveals that framing is widely researched within promoting health related behaviours (Reference: Digital Nudging: Altering user behavior in digital environments). However, the article states that little is known about framing and other psychological effect in digital nudging contexts, and questions whether the effects from “conventional” nudging are transferable. 

Message framing has been explored for several purposes, but it seems it has been most researched in the context of health messages related to persuasive systems and behavior change systems. There is a lack of research regarding digital nudging and message framing... However, since we are investigating digital nudging often comes in the form of a text message or notification, it is important to include the mentioned research when scrutinizing digital nudging as well. 

Burns et al. (2017) (Developing Persuasive Health Messages for a Behavior-Change Support-System That Promotes Physical Activity) investigates the topic of tailored persuasive messages that pertains to ergonomics. This research is still an ongoing project, where the referred article is the first of three parts. The experiment aims to examine the effects of tailored messages from a behaviour change support system that promotes physical activity. The user segment comprise participants in the age 18-30y, who have a sedentary lifestyle. The experiment entails three different types of messages, one focusing on the positive aspects of exercising (gain), and two on the negative aspects of not exercising (loss) with a difference in??. To develop and define a set of validated health messages was the goal of this particular study. Findings were that messages focusing on the benefits of being physically active (gain-framed) was the most persuasive for this user segment. The first loss-framed message that highlighted what a person could avoid by being physically active, was perceived as the second most persuasive. The second loss-framed message, pinpointed the consequences of not being regularly active, which had the least impact. The study was conducted as an online survey, and that the persuasive health messages were presented on a digital surface as such. The messages were not tested in a real life scenario, merly by feedback, i.e. it was not checked if the users actually responded in a physical manner (however this is hard to measure). The fact that this article was published in 2017, and that the findings offspring further studies, underpins the relevance of it. Though this study focuses on persuasive messages, and not nudging in particular, it is still relatable.

A recent article from 2019 (Reference: The effects of message framing characteristics on physical activity education: A systematic review) provided a systematic review on message framing in regards of physical activity. Effectiveness of messages was investigate for 13 studies with overall healthy adults as participants (one study with overweight) were included, based on quantitative data and statistics. All of the reviewed studies did test gain vs loss + one other message framing characteristic which varied, focusing on increasing the participation of physical activity. The findings conclude that gain-framed messages is most effective in the context of promoting physical activity, which is consistent with previous research. They mention that more research should be done on gain vs loss, including other message framing characteristics to strengthen the evidence of message framing and to expand the participants.
%This indicates the lack of qualitative research on the topic. Mention…  
The importance of this review lies in the fact that health/fitness professionals trying to influence physical activity behavior change should be able to use message framing to increase their effectiveness.


\subsection{How to nudge?}
For design of user interfaces in general there has been developed numerous of guidelines, principles and frameworks. Putting nudging elements into the user interface design must be done thoroughly. The user interface design will always have the ability to steer people in certain directions (Schneider, Weinmann and vom Brocke 2018 Digital Nudging:Guiding Online User Choices through Interface Design). What we choose is dependent on how the choices is presented. Due to limited cognitive ability, not always acting on rational behaviour, biases affect our decision making. Designers of behavior change systems (or other similar systems) has the ability to present the choice environment in a certain way to gently push users into a certain direction. 

What holds for step 3; design the nudge, is of deeper interest. Schneider further advise the designer to choose the suitable nudging mechanism in order to steer the users towards the defined goal / targeted decision. Even though the article mentions different tools and frameworks one can make use of for choosing mechanism. 

One of the first attempts at providing a framework for nudging in offline settings was the work by Ly et. al (A Practitioner’s Guide to Nudging) in 2013. It aims at making the process of implementing nudging more accessible, by suggesting a four step model; 1. map the context, 2. select the nudge, 3. identify the levers of nudging, 4. experiment and iterate. As we know, when computing became commonplace nudging was inherited to the development of information systems as well. In 2017, Meske and Potthof introduced the DINU model (reference: THE DINU-model—A process model for the design of nudges), a process model for implementing nudging in the field of Informations Systems (IS), which are strongly related to HCI. It propose a three phase process for designing digital nudges based on knowledge from the research area of nudging, but also from persuasive technology and theories of persuasion; analysing, designing and evaluating. 

Further, Schneider, Weinmann and vom Brocke, made suggestions on how to utilize the digital context of web sites and apps for implementing and designing nudges (Reference: Digital Nudging: Guiding Online User Choices through Interface Design, 2018). Their idea was that much of what applies to nudge designs in offline contexts can be inherited into digital contexts, as all choices are affected by human heuristics and biases. Furthermore, they emphasize that digital contexts not only consist of online consumer choices (an example they use a lot), but also systems, applications and websites related to e.g. health and social media. The proposal could therefore remind of the process that is followed when developing traditional/normal information systems and other applications; 1. define  the  goal, 2. understand  the  users, 3. design  the  nudge,  4. test  the  nudge. 

Using the work of Meske and Potthoff (2017) and Schneider et al. (2018) as inspiration, Mirsch et al. developed a new method for digital nudge design, that was empirically evaluated / validated and targeted for practitioners such as UX designers and similar (Reference: Making Digital Nudging Applicable: The Digital Nudge Design Method). They claim that the reason why implementation of digital nudging is nearly absent in practice, is due to lack of awareness and a systematic design approach for it.

More recently, Caraban et. al (Reference: 23 ways) enlighted the “how to nudge”-challenge from a more HCI perspective, by reviewing the current design space of nudging mediated through different technological prototypes. Through their work they concluded that even though existing frameworks (References: Beyond Nudge, Digital Nudging 2016 - Schneider, The Mindspace Way, Hansen and Jespersen 2013) brings valuable insights about cognitive biases and nudging mechanisms, they are not providing a complete answer for the “how to nudge” question. Nevertheless, Caraban’s own work contributes in connecting these two aspects, about which cognitive biases and which mechanism should be used to achieve the desired behaviour or choice. Their identification of nudging mechanisms serve as a framework for researchers and practitioners when developing and designing technology-mediated nudges. 

\subsection{How to measure the effect of nudging?}
Researchers and practitioners are asking how we should design effective nudges (reference: Functional Digital Nudges: Identifying Optimal Timing for Effective Behavior Change), but what is meant by effective nudges, and how do we measure them? Effectiveness refers to the degree to which the given intervention has persuasive ability, and there is a distinction between actual and perceived effectiveness (reference:, The Relationship Between the Perceived and Actual Effectiveness of Persuasive Messages: A Meta-Analysis With Implications for Formative Campaign Research). When we talk about perceived effectiveness, this is the user's own perception of the effect, i.e. the probability of the nudge to have a persuasive impact. Whereas actual effectiveness is the measure of the true impact the behavior or choice, which is done by applying quantitative research methods, collecting statistical data. The reason that many studies use or refer to perceived effectiveness is because it is difficult or almost impossible to measure actual effectiveness since behaviour is prone to several influencing factors. It is especially difficult to measure for digital nudging interventions aimed at physical activity promotion as would require activity meters. Besides, a quantitative research approach does not provide the full picture and understanding on why it worked or not. 
\subsection{When nudging fails}
Caraban et al. also discuss main factors to why nudging fails. One of them is the perception of intrusiveness. People may feel their autonomy is taken away.

\section{(Ethical) Implications and challenges}
The benefits one can gain from digital nudging and persuasive technology are apparently many and has gotten much attention. However, there are several ethical considerations to be aware of when implementing digital nudging. The lack of ethical considerations within research on persuasive technology is addressed by Torning and Oinas-Kukkonen (Reference: Persuasive system design : State of the art and future directions). In addition, studies scrutinizing nudging often mention these concerns, however they are claimed to only be grasping the surface of the problem. 

The definition of a nudge as Thaler and Sunstein (2008) defined it back in 2008, claims that it shall only be used for good incentives, i.e. for the users best interest (Reference: Nudge book). It is useful to question whether this is always the case. As with any other emerging technology, it may be exploited beyond its intentions. And even for the concern of increasing physical activity, is that really the best behaviour for every individual user? That discussion belongs to another discipline (philosophy) and we will leave it with that. 

\subsection{The fine line between persuasion and manipulation}
Through the review on digital nudging literature, both academic and business related, ambiguity is perceived about what constitutes as nudging and what is pure marketing. One of the nudge criteria is to safeguard the user's freedom to choose. It seems that the term nudging is often used incorrectly in business related contexts. 

By revisiting the categorization of nudges through Jespersen and Hansen framework (reference: nudge and the manipulation of choice), we see that nudges under the “automatic-non-transparent” category is said to “manipulate behaviour”. Such mechanism of nudging (e.g. subliminal priming, opt-out policies, adding inferior alternative, deceptive visualization, biasing the memory) is often unrecognized, and may therefore deceive users. This rise questions whether or not that category could actually embrace nudges, as users freedom to choose is not safeguarded. Examples of this is when nudging is used in commercial purposes, when the user is influenced to use money, take a choice that is not for the users best interest, but rather for the organization or company behind. Such nudge implementations are unfavorable and void way of implementing a nudge, due to its ethical implications. Transparency is important to avoid manipulation (Reference: Recommendations with a Nudge). 

\subsection{Criticism - Are they really that effective?}
Many studies praise the effect of nudging, and sometimes it seems that nudging is always a good idea. Some researchers (reference: Applying Behavioral Economics to Public Health Policy: Illustrative Examples and Promising Directions) (Matjasko et al. 2016) are more critical and questions the effects of nudging and discuss drawbacks and limitations. The first point of view is that the term nudging is vague and thin. (It should be noted that the article was published in 2016, and much has happened since then.) However, it also mentions that some interventions that goes under the category of nudging in fact should be described otherwise. Further, the authors speculate that nudges may lose effectiveness over time. It refers to studies where nudging in fact seems to have small and short lived effects. These findings contradicts what many other researchers believe, that nudges have a great potential. Research on the long-term effects of nudging is minimal and lacking to contradict these findings. Lastly, this article mentions that in order to become more successful, tailoring of nudging is needed. 

\section{Summary}
Through this literature review, it emerges that digital nudging is a broad concept with many underlying directions. It has also made us aware of the many needs and shortcomings in the research area. Firstly, a concise overview describing the various concepts (digital nudging, persuasive technology, behavior change systems and similar) and the connection between them should be developed to get a common understanding of the terms and how they relate.

Furthermore, it appears that the design aspect of digital nudges is extremely complex, thus a map describing the different paths within digital nudging is needed, in order to better understand which mechanisms and delivery methods (and other design aspects) that should be utilized in which situations. There are made some contributions i form of frameworks and methods on the process for designing of digital nudges. However, due to the novelty of the research field, few studies actually apply those methods and perform the design, implementation and evaluation of digital nudging, which emphasizes the need for it.

As with other technologies and strategies that aims to influence people's choices and behavior, digital nudging could be applied in physical activity promotion. Government agencies (around the world) are demanding good interventions to reduce the number of inactive citizens.

The framing effect is a psychological effect that has been extensively studied in the field of nudging in offline settings, but also within variations of persuasive design. Statistics from existing research show that gain-framed messages have the greatest impact to achieve targeted behavior when it comes to preventative behavior, such as physical activity. However, the lack of qualitative research scrutinizing users perception and experience regarding this imply a incomplete understanding. 
