\chapter{Related Work}

%Here you introduce related work, such as 
\cite{acquisti_nudges_2017}, \cite{al_stairs_nodate} \cite{suri_stairs_2014}, \cite{fogg_persuasive_2003}, and \cite{hamper_behavior_2016}.
%Who else has done work with relevance to this work of yours? 
%What did they do? 
%What did they find? 
%And how is your work here different

%This master thesis had to adapt aims, objectives and research question after the pandemic hit Norway and caused problems with implementation of the study as it was originally designed. This chapter is strongly affected by this, and does therefore include some theory on topics (e.g. message framing) that are not elaborated on or discussed any further. 

\cite{karl_chapter_nodate}

\section*{Preface}

%\colorbox{yellow}{Texthheiejrio}

The society and world we live in is constantly subject to digitalization, and mankind use technology for many of the daily tasks and choices. The choices may have many appearances and forms, e.g. whether you should book the annual dental check soon, go to that spinning class later tonight or what to make for dinner, technology has proven to be effective in such ways. Rational thinkers may say: just book that appointment, you should definitely go to that spinning class because it makes you feel good and maybe you should try that salmon with vegetables tonight instead of the usual macaroni and cheese. However, we are not always rational thinkers and must admit that we often stick to bad habits and behavioural patterns, predominantly due to our tendency to choose the path of least resistance. 

Luckily, some of the technology surrounding us is actually designed to steer us to make better choices. It’s both time consuming, effortful and exhausting to be reflective all the time, most would claim. It must feel good to to have a guiding hand, knowing which decision which benefits you the most in certain situations? The transition is happening now, as a result of new findings within the field of designing for behavioural change. 

Behavioural design aims to shape human behaviour in a number of disciplines; architecture, furniture, public transport, urban planning, etc (Wendel, S. (2014). Designing for behavior change : Applying psychology and behavioral economics. Sebastopol, Calif: O'Reilly). (It is undoubtedly right to assume that behavioral design also could have an impact on health related issues.) As computing became commonplace, another direction was introduced as a sub-group of behavioural design, namely persuasive design, and more specifically persuasive technology. Which hinges around designing computers to computers ability to change people's behaviour.  However, in some situations, the aim is however not to \textit{change} a certain behaviour, but rather to influence the decision-making process towards \textit{maintaining} a behaviour. Nudging is an example of this. Both digital nudging and persuasive technology is two design strategies/principles with intention… 

%Noe mer om persuasive technology og nudging? Hvordan binde de sammen?



\section{Health}
Health is a prominent application domain and area of interest within the human-computer interaction field, both when it comes to health informatics, healthcare technolgies and personal informatics for health and wellbeing (Reference: Technological Approaches to Promoting Physical Activity, HCI and mobile health interventions, HCI for health and wellbeing: Challenges and opportunities). There lays much potential for motivation, encouragement and the ability to change people’s health-related behavior through user interface design as people are becoming rapidly more dependent on technological/digital interfaces. Over the last decades, the development of health technology that promotes or helps people to physical activity has evolved from completely primitive step counters to wearable fitness trackers, often with a corresponding persuasive mobile application to visualize data. Now, the training apps sometimes are integrated with both activity tracker devices  such as heart rate device, smart watches and gym membership. 

Evolution of technology for different health promotion started of with elementary step counters, large, clumsy artifacts that could be perceived as disruptive in everyday activities, to become step count as the most primitive functions that is embedded in almost all ubiquitous and wearable devices nowadays. From that to fitness trackers, smart and pulse watches, to persuasive mobile application and other health wearables, integrated gym app.

Visualizing, tracking, monitoring and logging activity data motivates people to train more (reference). More recent design principles such as gamification, feedback, rewards, challenges/competitions and social influence has been embedded into the fitness devices. (Reference: The Potential Role of Digital Nudging in the Digital Transformation of the Healthcare Industry)
 
Around the peak in persuasive research, several articles aimed to give an overview of the technological approaches in different health related context. One article that mentions current and potential contributions in promoting physical activity at the time being was conducted by Maitland and Siek (Reference: Technological Approaches to Promoting Physical Activity). Monitoring capabilities and activity inference techniques. Effective user experience. Applications that provide Monitoring of physical activity often combine it with other different methods / techniques to encourage users to be more active/behavior change, for instance through goal setting, feedback and social influence.  The article states that physical activity promotion and support systems has the power to lower the barriers to physical training, but cannot solve this problem alone. Health education is one of the other ways in solving this problem, at least in certain parts of the world. 

\subsection{Barriers to Physical Activity} 
In order to develop and define guidelines for nudging mechanisms regarding physical activity promotion // understand how we can design effective nudges in digital contexts to promote physical activity, one need to visit the barriers which prevent or counteract physical activity. Such barriers are strongly dependent on several demographic parameters; age, gender, geographical location, nationality, cultural background, educational level, literacy, social and economic class and other (Reference:Perceived barriers to physical activity across Norwegian adult age groups, gender and stages of change - ok å bruke den kilden? eller må man egentlig ha kilde på det?) . Due to this, physical activity barriers are often researched within a limited scope, and with very specific user segments, such as diabetics, elderly, pregnant, college students, etc. However, it is reasonable to evaluate lack of motivation, information or time (hva mener jeg her?), and injuries or disabilities, as “general” barriers for accomplishing physical activity independent of age and geographical location (Reference physical barriers). Both lack of motivation and information (are examples where technology to motivate and educate is utilized) are manageable to overcome, because one can educate or motivate people through nudging. People suffering from serious diseases will probably not benefit from nudging because motivation is hardly a factor. However, for people that suffers from motoric diseases, chronic conditions or similar disorders that require self-management/care tools, could for sure be assisted by nudging (Rouyard et al. 2018). This leads us over to next section... 

\section{Persuasive Technology}
As mentioned in the introduction, a concept that is popular and accentuate nowadays, in all fields of informatics, design and psychology, is persuasive technology. Within this collective term there exists a numerous applications, systems and softwares, which intentions are to persuade users to act on something. That something could for instance be to stop smoking, drink more water or avoid snacking. Persuasive technology aims to change people’s behaviour, attitude or habits towards the most beneficial, through interactive systems (Orji and Moffatt 2018 - Persuasive technology for health and wellness: State-of-the-art and emerging trends). Based on a various of strategies and principles, the interface is designed to assist the user to gain motivation or interest for the sought behaviour change in a given context. This research field was introduced around 1990, by one of the discipline pioneers, B.J Fogg, a behavioral psychologist. His contribution became the very foundation for further research and proceedings. He stated that the power of designing for behavior change, should only be used towards enhancing society and people's life. 

People are situated amongst and exposed to persuasive technology everywhere in the society, without necessarily being aware of it. The most typical application domains are education, sales, religion, sustainability, insurance, military and public health. The latter, persuasive technology influencing health and wellness, has received vast attention and lately expanded on its targets (Orji and Moffatt 2018) (Persuasive technology for health and wellness: State-of-the-art and emerging trends). For what pertains to health and wellness, two main directions have evolved; 1. disease management and 2. health promotion / prevention. Disease management hinges around enhanced self-monitoring and management for individuals that has already incurred an illness, injury or disease. It could for instance be self care technology for diabetics (Rouyard et al. 2018) or self medication to cure HIV (Schnall et al. 2015). As for the other direction, i.e. persuasive technology for health promotion or prevention, it addressed the proactive and inhibitive aspects of physical and mental health. It could prevent diseases through promotion for physical activity or healthy dieting. It can also be used to encourage self diagnostics, e.g. self monitoring of mole cancer, breast cancer or similar, to detect symptoms at earliest  possible stage. In this thesis, I will mainly focus on health promotion and prevention, specifically physical activity promotion. 
\begin{itemize}
\item More on Foggs behaviour model: In order to achieve an certain behaviour, three elements must be present at the same time: motivation, ability and trigger.  
\item  Fear 
\item Framing
\end{itemize}

It should be noted that the persuasive technology and design sphere in the health domain embraces several categories. There exist a numerous of different devices, ways of implementations, user interfaces designs, e.g. step counter, fitness trackers and other wearables, exergames etc (Reference: Technological Approaches to Promoting Physical Activity, 2009). However, this review will be limited to concern the physical activity domain and PT within mobile applications, as these frames the task. 

\subsubsection{Persuasive design for Physical Activity}
According to authors X, motivation is a powerful factor when it comes to physical activity promotion. One study that underlies this (Behavior Change Support for Physical Activity Promotion: A Theoretical View on Mobile Health and Fitness Applications) views mobile applications for behaviour change. Motivation could be both intrinsic or extrinsic.  

(Reference: Persuasive technology for health and wellness: State-of-the-art and emerging trends, 2018).- denne handler om mange tema innenfor helse, ikke bare PA. 
The review by authors x showed that physical activity was the most researched amongst different health domains. They were utilizing different theories from behavioural science (Transtheoretical Model, Goal Setting Theory, Theory of Meanings of Behavior, Self-determination theory, Operant Conditioning) and motivational strategies (Tracking and monitoring, Social support, sharing, and comparison,  Competition, leaderboard, ranking, Reward, points, credits,  Goal and Objectives,Audio, Visual and Textual Feedback).  
Where motivational strategies such as tracking and monitoring was amongst the most researched. Whilst persuasive messages, reminder, (which could be similar to the digital nudge interface)and alert where not identifies in the research of physical activity.
They reviewed the effectiveness of diffrent PT for physical acitivty. 
Gjennom 16 års arbeid kom det frem 85 relevante artikler som de så på. Kun 15 prosent av disse fulgte en kvaliatativ forskning tilnærming. Uavhening av hvilket helsetema de handlet om viser at det er få studier rundt brukeropplevelse og oppfattelse av PT for physical acitivty promotion. 

(Reference: Persuasive Technology in Mobile Applications Promoting Physical Activity: a Systematic Review, 2016) - Denne handler bare om PA, 10 år. Følger PSD for å evalurere artikler som omhandler tema. 
"A combination of praise, rewards, reminders and suggestion were used to motivate the user to achieve their goals. And yet, many applications do not take advantage of well-known and proven features. For instance, previous research has identified reminders as an important category for users to continue adherence to targeted behavior [46, 47], however, not all reviewed research included the reminders as a persuasive feature. "

It is known that PT whit features of system credibility will achieve higher level persuasiveness. (Does this hold for digital nudging as well?) 
Dialog support features (such as reminders): minner om digitale nudger? Nudger kan bli sett på som en del av dialog support feature. Eller kan sammenlignes med de. 
Konkluderte med ate ulike persuaisve design guidelines / principles (28 fra PSD) er avheningac hverandre. 

(Reference: Persuasive applications for the healthy lifestyle, 2017) 

\subsection{Design Principles}
During the years of research on persuasion through technology (three last decades?) , different guidelines, principles and frameworks have been evolved. The study by Oinas-Kukkonen and Harjumaa (Reference: PSD: Key Issues, Process Model, and System Features), has in particular been recognized and widely referred to in the field persuasive technology. It provided a framework for designing and evaluating persuasive systems. Four dimensions of support was identified; primary task, social support, dialog support and system credibility. Within these dimensions, 28 underlying design principles were announced. 

Tailoring are among the methods that have been investigated in relation to persuasion (Reference: Persuasive system design : State of the art and future directions) - og hva ble funnet ut?

\subsubsection{Tailoring}
One of the design principles is tailoring, which have been widely investigated (Reference: Persuasive system design : State of the art and future directions). It belongs under the dialog support dimension, and refers to a systems ability to achieve higher level of persuasiveness when content and services are tailored to personal factors such as needs and characteristics, or contextual factors such as usage. 

It should be mentioned that the literature 
Når det kommer persuasive wearables ... 
Men vi fokuserer på: persuasive mobile applications 

%\subsubsection{%Personalizing}
%Another common, and quite similar design principle from the PSD model is personalization (PSD: key issues...). 
%The main difference from tailoring, is that personalization is more granulated / of higher resolution by focusing on relevant content and services for each individual, and not as a user group. It is granted that personalizing of content in persuasive technologies, has great impact in order to change people's behaviour (reference). However, it raises new concerns towards privacy which will be expanded on later on (in the section about Ethical Concerns). Several discoveries have been made on the effect of personalization within persuasive technology, user interface design, personal informatics (PI) and technologies for behaviour change (BCT). Making the content more geared to the individual user has seen to gain user engagement. However, to keep user engagement lasting  and durable is a challenge regarding such technologies, as users tend to lose some interest after period of time (Caraban et al. 2019).


However, to keep user engagement lasting  and durable is a challenge regarding such technologies, as users tend to lose some interest after period of time (Caraban et al. 2019).
%The process of personalization is known by many names and is referred to as both customization and individualization in previous literature. 

These terms implies tailoring of software, web interfaces, mobile applications and other digital services, to benefit the user (Treiblmaier and Pollach, n.d.) (Reference: Users' Perceptions of Benefits and Costs of Personalization, 2007). It means that the content and the way that it is presented will adapt for the individual user based on userdata. Often this data is information about country, language, location, etc. Using such data to personalize the service or application, are design principles that almost every commercial company take advantage of today. Online streaming services, grocery shops, travel agents etc does it, to different degrees. Comercial players are using these elements as effectively as they possibly can within the boundaries of technology and legislation. Thus health promotion initiatives involving nudging may have benefits of using same. Also, in the health sector there is a fine line between what can perceived as commercial interest and libertarian paternalism. What are the motivation to implement it in any case? To get more engaged users, less hassle and time consuming, increase effectiveness. 

Personalization of content is an encompassing design principle that embraces many aspect. In the context of digital nudging is is shown that: 
And for physical activity / health promotion in other persuasive technologies it is shown that: 
Personalization can be implemented at many stages and levels of a technology, application or system. It could be anything from Netflix suggesting movies that you are more likely to enjoy based on your feedback or rating from other sessions. It could be that an app is tracking your location and will recommend you to try a certain cafe. These examples are quite advanced. Personalization could also be based on more simple characteristics such as your age or gender, e.g. technology using your name to give a sense of social communication. 

In fact, how we communicate is a powerful factor when it comes to technologies for behaviour change and nudging interventions (both offline and digital), since it often boils down to how the possible choices and directions is presented. One of the most mentioned communication strategies within this research area is framing, which we will elaborate on later on. 

\subsubsection{Hvorfor her? Message Framing}



\subsection{Psychology behind}
What makes persuasive technology so powerful? One of the answers in order to manage and understand this, is the fact that it builds upon theories from both psychology, cognitive science and behavioral economics. Together with developed design features and principles within persuasive design, it underlie how the technology is experienced and perceived as convincing. Firstly, I will describe some of the essential details from psychology that aims to explain how people think, and then some design principles and motivational strategies, in the following paragraphs. 

Cognitive dissonance and guilt

Fear

\subsection{Communication}
Whenever technology is used as a medium for persuasion, it becomes natural to visit the aspect of intentional communication (Reference: Persuasive System Design: State of the Art and Future Directions). 

Fields of computer-mediated communication, rhetoric and psychology are strongly related to the HCI community, as they overlap and complement each other in the "How to design" questions. 

%Is the aim always to persuade? Is there different ways to persuade? Short-term and long-term? Is it about convincing the small and direct choices (here and now), such as whether you should go to training today or not. Or the long-term choices, to maintain a healthy and active lifestyle. 

Although one can say that research on persuasive technologies and communication had its heyday around 2009, it has remained an important topic and elaborates into new directions. 


\section{Digital Nudging}
In recent years, a new strategy for design with intention has emerged in the siding of persuasive design; digital nudging. The concept refers to influencing people behavior dada through purposeful .... application of user-interface (UI) design elements in digital environments, such as... 

Mange likheter med bahaborial desigm... 

It origins from the term nudging within behavioural economics, ans was coined in 2008 by Thaler and Sunstein. They defined the term as,...

Gradually, the nudging techniques were passed on to digital interfaces, hence digital nudging. It is therefore quite new...  research field... 

Based on reviewed literature it seems that it is first discussed around 2016, but since there are several concepts / cases that can be considered as digital nudging (such as prompts, reminders, push notifications, etc), it is difficult to confirm precisely. In fact, digital nudging often appears in practise, without consciously applying digital nudging strategies during design and implementation (Reference: Digital nudging altering user behavior in digital environments). You may have noticed them without being aware of it, small subtle changes in the choice environment on your phone or at the web. Supposedly they are everywhere around us.

As digital nudging is partly overlapping with principles of persuasive and behavioural design, it can be hard to distinguish between them (whether this is desired at all?). It appears that no successful attempt at describing the real relationship between those two has yet been made. 

Findings and insights from studies on, for example, persuasive technologies, communications and systems, can be passed on for digital nudging.

In the perspective of digital nudging, it is used to design user interfaces to change people behaviour. 

It is a novel, yet prosperous, research area, touching upon several fields of expertise. Many of our daily basis choices can be made on a digital interfaces and devices. Smart phones, laptops, tablets and other technological devices are everywhere around us, whether we are at work, school, home or on the go. The unlimited access to screens and other technological widgets has made mankind dependent on, and in many cases succumbed to, such devices (Digital Nudging: Altering User Behavior in Digital Environments). This also implies that this type of user interface is powerful and can influence people in certain directions if designed thoroughly. %Digital nudging may be a novel research area, but it builds on the comprehensive research from behavioral economics and social psychology, where conventional / traditional nudging is scrutinized. 

\subsection{Digital choice environments}
Digital choice environments was mentioned above, and refers to .... 
This study will further concern mobile applications and not warbles etc... 

\subsection{Nudging}
%To further understand the effects of digital nudging, it could be useful to scrutinize its origin. Conventional nudging is a legacy and new understanding of behavioural design, arising from behavioural economics (Reference: Nudge book). 

Digital nudging may be a novel research area, but it builds on the comprehensive research from behavioral economics and social psychology, where conventional / traditional nudging is scrutinized. Some of the most known and widely discussed examples of nudging are those which are implemented in offline contexts / on tangible(?) surfaces. 

%URELEVANT? For instance, a fly was painted on pissoirs so that the users wanted to aim it, thus avoiding spills with 80 percent at Schipol Airport (Reference: Nudge Book). 

For the same reason, steps were painted at the staircase so that people would choose the stairs instead of the elevator (reference). 

%URELEVANT? In countries were being organ donor is opt-in the percentage is quite low (15), but as soon as you make this choice opt-out, then the number of donors is higher than ever, actually over 90 percent of the population (Reference: Caraban et al. 2019). This illustrates how people tend to go with the path of least resistance. 

However, the definition of a nudge can still be a bit confusing and vague, cause there is a numerous of different appearances for digital nudging. And different facets, which we will explain later in this chapter. 

\subsection{The Dual Process Theory}
Human decision making is compounded process. The Dual Process theories of decision-making is an important contribution in order to understand the human behavior (Caraban et al. 2019)(reference:  Thinking Fast and Slow). Humans have two modes of thinking: the automatic and the reflective. The reflective system is based on rational processes and shaped by consciousness. This means that it is slow, effortful and goal-oriented. According to reference, most of HCI research (94 percent) focuses on this way of thinking when exploiting the effects of nudging. Nevertheless, it is the automatic system which is used in most cases during a day (95 percent). It is responsible for repeated and skilled actions, and works therefore quick, needs minimal effort, and is based on instinctive, emotional.  unconsciously).Including research on the automatic system in the context of nudging, is referred to as an untapped areas of opportunity.

Humans are irrational and tend to take choices accordingly. Sometimes we need6
a gently push in a certain direction, to take more sustainable, healthy and expedient choices for ourselves and the society. Inactivity and public health is a worldwide problem  and  is  often  used  as  the  purpose  to  study  the  effects  of  nudges.   Most  studies regarding physical activity promotion exploits the reflective mind (reference), whilst the idea of being able to promote motivation for physical activity to be handled by the automatic system could be revolutionary. In fact,  the literature rise questions if technologies can rely less on users will and capacity to engage, meaning exploiting the automatic system.

\subsection{Biases and heuristics }
Other theories from psychology, such as biases and heuristics are key concepts within this research field, as they are able steer people's decision making. Human forecast is proven many times to be biased, and our decisions is inveigled by biases and flawlessness. Cognitive biases refer to repeated missteps in how we think (Reference: Nudge Book). In other words, we make unreasonable or erroneous conclusions based on things we experience or personal dispositions. This can lead us to make irrational choices. Humans are affected by such on everyday basis. There exist several types of cognitive biases, and the most utilized and referred to in the context of nudging is status quo and loss aversion. Nudging can help the individual to make optimal choices by detaching from systematic cognitive biases (Reference: Nudge Book). 

\subsubsection{Loss Aversion Bias}
The loss aversion bias is based in that there are stronger emotions associated with a loss than a gain. This means we would do very much to avoid losing something. If we know we are about to lose something, we become impervious. We will not accept losing something.

Nudging can help the individual to make optimal choices by detaching from systematic cognitive biases (Reference: Nudge book). 

%\subsection{}
\subsection{How to nudge?}
There are surprisingly low number of articles which appears when submitting the search query including both digital nudging and design, which amplify its novelty, and the need for more research on this topic. The suggestions on design methods and models for implementing nudging will be reviewed in the following paragraphs. 

The first attempt at providing a framework for nudging in offline settings was the work by Ly et. al (A Practitioner’s Guide to Nudging) in 2013. It aims at making the process of implementing nudging more accessible, by suggesting a four stepped model; 1. map the context, 2. select the nudge, 3. identify the levers of nudging, 4. experiment and iterate. %However, the model concentrates on nudging in offline settings.

As computing became commonplace... nudging was inherited to the development of in formations systems. 

In 2017, Meske and Potthof introduced the DINU model (THE DINU-model—A process model for the design of nudges), a process model for implementing nudging in the field of Informations Systems (IS), which are strongly related to HCI. It propose a three phase process for designing digital nudges based on knowledge from the research area of nudging, but also from persuasive technology and theories of persuasion; analysing, designing and evaluating. However, this suggestion remains conceptual, as the proposed model is not evaluated yet and the research is still in progress.

Further, Schneider, Weinmann and vom Brocke, made suggestions on how to utilize the digital context of web sites and apps for implementing and designing nudges (Reference: Digital Nudging: Guiding Online User Choices through Interface Design, 2018). Their idea was that much of what applies to nudge designs in offline contexts can be inherited into digital contexts, as all choices are affected by human heuristics and biases. Furthermore, they emphasize that digital contexts not only consist of online consumer choices (an example they use a lot), but also systems, applications and websites related to e.g. health and social media. The proposal could therefore remind of the process that is followed when developing traditional/normal information systems and other applications; 1. define  the  goal, 2. understand  the  users, 3. design  the  nudge,  4. test  the  nudge. 

Using the work of Meske and Potthoff (2017) and Schneider et al. (2018) as inspiration, Mirsch et al. developed a new method for digital nudge design, that was empirically evaluated / validated and tailored for practitioners such as UX designers and similar (Reference: Making Digital Nudging Applicable: The Digital Nudge Design Method). They claim that the reason why implementation of digital nudging is nearly absent in practice, is due to lack of awareness and a systematic design approach for it.

More recently, Caraban et. al (Reference: 23 ways) enlighted the “how to nudge”-challenge from a more HCI perspective, by reviewing the current design space of nudging mediated through different technological prototypes. Through their work they concluded that even though existing frameworks (References: Beyond Nudge, Digital Nudging 2016 - Shcenieder, The Mindspace Way, Hansen and Jespersen 2013) brings valuable insights about cognitive biases and nudging mechanisms, they are not providing a complete answer for the “how to nudge” question. Nevertheless, Caraban’s own work contributes in connecting these two aspects, about which cognitive biases and which mechanism should be used to achieve the desired behaviour or choice. Their identification of nudging mechanisms serve as a framework for researchers and practitioners when developing and designing technology-mediated nudges. %They suggest that future work should evolve frameworks and tools to support the nudge design area, based on their findings??

%What we have so far: different types of nudging mechanisms that can be broadcasted through different channels or implemented in different ways, as a part of different digital choice environments. (Hører dette til diskusjon?)

\subsection{The many facets of Digital Nudging}
Some/all of the aforementioned nudge design methods and models include a phase in which to select the proper nudging mechanism. This leads us into the challenging journey through the possible ways of digital nudge implementation. Through the systematic review on technology mediated nudging by Caraban et al. (Reference: 23 ways), 23 distinct mechanisms for nudging was identified. The mechanisms where classified in six different categories:

\begin{enumerate}
\item The facilitate nudging mechanism, reduces people's physical or cognitive effort, hence facilitating for decision making. It is based on the status-quo bias which refers to that humans tend to go with the option with least resistance. One good example is that if a choice is already made for us, we often stick to this, because the process of taking the choice individually requires effort, time and consciousness. This could typically be a default or opt-out setting in the choice architecture. (Also positioning, hiding and suggesting goes under this categorization.) 
\item Confront nudges introduce doubt when people are about to make a decision. It exploits the regret aversion bias, which means that people are extra careful in a situations where risk is perceived. By breaking mindless behavior and make suggest using the reflective mind. 
\item A deceive nudging mechanism adds inferior choices so that a particular choice is promoted harder than the others. 
\item The social influence nudge takes advantage of what people believe is expected from them, such as adapt to or accept certain norms or social rules. 
\item Nudges in the fear category, uses the feeling of fear or loss to make... 
\item Reinforce: repeat actions to learn reflective thinking. 
\end{enumerate}

Further, Caraban et. al (2019) makes an attempt to define in which situations the different nudging mechanisms should be used. Through Fogg's behavior model they are able to categorize all the 23 nudging mechanisms in one of the three forms: facilitator, signals and sparks.
\begin{itemize}
\item Facilitator: (ability strategies), makes an task or a choice easier for users. Reducing distance and effort. 
\item Spark: (motivational strategies), says to be optimal in situations where the user lack motivation 
\item Signal: (trigger strategies), optimal in situations where the is a divergence between a users intention and actions, but where motivation and ability is often present. 
\end{itemize}

In order for nudges, belonging to the category of signals, to result in the achieved targeted  behavior or decision, some variables should especially be considered. Those variables are timing, frequency and tailoring (Reference: 23 ways). A signal should appear in the right moment in order to be successful, however, what characterizes the right moment will depend on the situation and purpose of the nudge. Caraban et al. also claims that frequency may affect the effectiveness of the signal. In addition, another study shows that tailored signals / prompts / reminders / nudges with higher frequency are more effective than generic prompts with rarer frequency (Reference: Periodic Prompts and Reminders in Health Promotion and Health Behavior Interventions: Systematic Review). Within tailoring, it also appears different variables (for instance content and framing?), which leads us on to the common design principle of persuasive design. Which we are describing later on.  

Despite the research already covered on the topic, an complete overview of the different types and implementations of digital nudging is still missing. Other researchers have categorized nudging in a different approach, sorting them within other types of labels. For example, Jespersen and Hansen, defined four different classes where they categorised the nudges in (Reference: Nudge and the Manipulation of Choice. A Framework for the Responsible Use of Nudge Approach to Behaviour Change in Public Policy). Two questions determines their four classes: if the nudge is aimed at the reflective or the automatic mind, and if the nudge is visible for the user or not (also called transparency). This creates four intentions of nudges. 
\begin{itemize}
\item Automatic-transparent: nudges that intend to influence behavior (e.g., changing the default option)
\item Reflective-transparent: nudges that intent to prompt reflective choice (e.g. the“look right” painted on the streets of London)
\item Reflective-non-transparent: nudges that intent to manipulate choice (e.g., adding irrelevant alternatives to the set of choices with the goal of increasing the perceived value of certain choices)
\item Automatic-non-transparent: nudges that intent to manipulate behavior (e.g., rearranging the cafeteria to emphasize healthy items)
\end{itemize}

Since different categories and labels are used on seemingly similar nudging mechanisms, it seems that there is a need of a structured overview of the digital nudging hierarchy. 

%Prøve å forklare hierakiet og hvordan alt henger sammen 

\subsection{Communication Channels/Digital form of delivery}
%As mentioned above, there are different types of nudges, and they come in different forms. In the digital sphere, nudges are often presented through e-mail, sms, push notification or integrated in a mobile application (reference: Digital nudges and dark patterns: The angels and the archfiends of digital communication).

%Although the use of push notifications is often associated with spam and commercial content (reference?), this is also a channel that digital nudging utilize. There is a lack of literature on this in specific. 

Another aspect of digital nudges is the technology channel they are mediated through. Mirsch et al. mentions the choice of technology channel to deliver the intervention as a part of the analysing phase, in their proposed method for digital nudge design (Reference: Making Digital Nudging Applicable: The Digital Nudge Design Method). They describe web sites and mobile applications as two examples of a technology channels/digital environments. The chosen technology channel should be familiar to its audience. 

There are different methods for delivery within a digital context/environments or technological channel, such as smartphone or fitness trackers (Reference: Functional Digital Nudges: Identifying Optimal Timing for Effective Behavior Change). SMS, email, push notification or vibrations are some of them. Authors X discuss that more investigations are needed to establish which delivery method should be utilized for a specific target behavior. 

Even though several studies mentions push notifications as a common way to deliver digital nudges, there are few studies which scrutinize this in relation to user experience and perception. 
%Isn't push notifications often associated with interruptions,...?
%spam and commercial content(?)
%FINNE UT STUDIER PÅ PUSH NOTIFICATIONS 

Other considerations is as mentioned timing and frequency, however, this was not the focus for this thesis.  

\subsection{Ethical implications}
The benefits one can gain from nudging (and persuasive technology) are apparently many and has gotten much attention.  However, there are several ethical considerations to be aware of when implementing nudge mechanism into the (digital) choice architecture. Studies scrutinizing nudging often mention these concerns. As Thaler and Sunstein (2008) defined a nudge, it shall only be used for good incentives. It is useful to question whether this is always the case. As with any other emerging technology, it may be exploited beyond its intentions. 

In 2009, the lack of ethical considerations within research on persuasive technology was addressed (Reference: Persuasive system design : State of the art and future directions) - har det blitt bedre? 

\subsubsection{Lack of Transparency}
By revisiting the categorization of nudges done by Jespersen and Hansen, we see that nudges under the “reflective-non-transparent” category is said to “manipulate behavior”. This rise questions whether or not it could actually be called a nudge. This mechanism of nudging is non-transparent, and may therefore deceive users. Examples of this is when nudging is used in commercial purposes, when the user is influenced to use money, take a choice that is not in best interest, but rather for the organization or company behind. The 23 article does actually not categorise any of the defined nudging mechanism in this particular category, which reflects that this is an unfavorable and void way of implementing a nudge. 

\subsubsection{Exploitation}
The fact that nudges are helpful and useful, makes them exposed to be exploited by choice architects and designers with other intentions than what is desired. Often the people behind the implemented nudges and the choice architecture want to benefit themself and their employers, rather than the users. There will always be a risk and uncertainty that this happens (Reference: Nudge book). Domains where nudging is used for non-beneficial purposes for the benefit of the individual, society and the environment, could be advertisement, marketing, e-commerce, online retailing and others. It’s interesting to have a look on how the business articles as well, det står mer der om dette enn i forskning for eksempel. 

\subsubsection{Criticism}
Many studies praise the effect of nudging, and sometimes it seems that nudging is always a good idea. However the article Applying Behavioral Economics to Public Health Policy: Illustrative Examples and Promising Directions (Matjasko et al. 2016) questions the effects of nudging and discuss drawbacks and limitations. The first point of view is that the term nudging is vague and thin. It should be noted that the article was published in 2016, and much has happened since then. However, it also mentions that some interventions that goes under the category of nudging in fact should be described otherwise. Further, the authors speculate that nudges may lose effectiveness over time. It refers to studies where nudging in fact seems to have small and short lived effects. These findings contradicts what many other researchers believe, that nudges have a great potential. Research on the long-term effects of nudging is minimal and lacking to contradict these findings. Lastly, this article mentions that in order to become more successful, tailoring of nudging is needed. 

And some raise criticism against the beviset på om de faktisk fungerer. Siden det er vanskelig å måle oppnådd effekt av digital nudging. 

\subsubsection{The fine line between persuasion and manipulation }
Through my review on existing literature, I experience ambiguity about what constitutes as nudging and what is pure marketing. One of the nudge criteria is to safeguard the user's freedom to choose. It seems that the term nudging is often used incorrectly in business related contexts.

\section{Summary}
%Notater
The obvious after reviewing existing literature in the field is that there is a lack of user experience of digital nudging. Many articles mention that nudging often comes in the form of push notifications in digital contexts, but studies that deal with this form of digital nudging are are almost absent in the literature. The lack of an concise overview of different types of nudges emphasizes the need to examine the different types more closely. 
%Much of the literature is based on psychology aspects without really testing them (?). 

The literature review enlightens several needs. First of all, there is a lack of an overview that describes the connection between the different concepts.

Når det kommer til forskning på digital nudge intervetions, handler de fleste om å måle effektivenessen av interventionenem, som gjøres gjennom kvantitativ data og statistiske analyser, ved å faktisk måle om nudgen førte til et spesifikt valg (Reference: Making Healthy Choices Easier: An Exploratory Study of Nudging Interventions Across Germany and Denmark + other articles?

What we have in regards of digital nudge design so far is different types of nudging mechanisms that can be broadcasted through different technology channels, as a part of different digital choice environments.