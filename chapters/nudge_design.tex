\chapter{Digital Nudge Intervention Design}
This chapter describes the process of designing, organizing and implementing the digital nudge intervention that was implemented and tested in a persuasive gym app context with real users for this study. 
%This chapter is devoted to describe the design process of the digital nudges that are investigated in this research. 

\section{Method} 
%Explain how the digital nudges that were tested were designed:
%Also got inspiration from earlier studies, conducting similar studies. 
As stated in the Chapter 2, there has been some contributions into the design of digital nudges, in regards of process, methods and frameworks. For the development of the digital nudges implemented in the Sats app, we partially followed the proposed DND method of Mirsch et al.\cite{mirsch_making_2018}. The reason why we chose it was because it is the currently most defined method to follow for designing digital nudges. However, as it constitutes a method for the overall/complete development of digital nudges and in light of the organizational goals, and we were only supposed to add a digital nudge to an already existing app, it was only used for inspiration and guiding, meaning that is was partly applied. 

The DND method consist of three phases: analysing, designing and evaluating. As with any other design process, the last phase suggest that evaluation should be done to measure if the implementation does fulfill the desired effects. If not, one should go back the first phase. However, due to the scope of this research we are not performing this as an iterative process. 

An overview of what was done in the different phases of development will follow. 

\bigbreak

\textbf{Phase 1 - Analysing}

\bigbreak
\begin{figure}
\includegraphics[width=1\textwidth]{images/Phase1.png}
\caption{"Phase 1: Analysing" in Digital Nudge Design method by Mirch et al. \cite{mirsch_making_2018}.}
\end{figure}
\bigbreak

\textbf{ R1:} As the digital nudge intervention was implemented in collaboration with Sats, and their already existing app, this requirement was not applicable. However, we assume that the organizational goals of Sats are to keep members active and aim to help members achieve the national recommendations on physical activity.

\textbf{R2: }The targeted behaviour is regular physical activity. However, we do not aim to measure the effectiveness of the intervention, i.e. whether the targeted behavior is met or not is not the main focus as we rather investigate it qualitatively. 

\textbf{R3:} The users (i.e. the participants)  have different goals, but they share the intention to be physical active at some level. Specifying the different goals is beyond the scope of this task.

\textbf{R4: }Users characteristics varies in age, gender, type of motivation, attitude and interest in physical activity. Identifying all these are also without the scope of this task. (However, some user groups were defined in Chapter 3).

\textbf{R5:} The collaboration with Sats allowed us to nudge users through push notifications in the Sats app. In practise, SMS or email could also be used. 
\bigbreak
\textbf{Phase 2 - Design }
\begin{figure}[h]
\includegraphics[width=1\textwidth]{images/Phase2.png}
\caption{"Phase 2: Design" in Digital Nudge Design method by Mirch et al. \cite{mirsch_making_2018}.}
\end{figure}
\bigbreak
\textbf{R6:} Message framing was in applied in order to investigate its effect in this context. 

\textbf{R7:} The intervention was designed to disseminate health information that was randomly broadcasted during 30 days. 

\textbf{R8:} The designed intervention was inspired by similar studies of persuasive messages and technologies for physical activity promotion. 
 \bigbreak
\textbf{Phase 3 - Implementation and Evaluation }
\bigbreak
\begin{figure}[ht]
\includegraphics[width=1\textwidth]{images/Phase3.png}
\caption{"Phase 3: Implementation and Evaluation" in Design in Digital Nudge Design method by Mirch et al. \cite{mirsch_making_2018}. }
\end{figure}
\bigbreak
\textbf{R9:} The digital nudge intervention was implemented in the defined technology and evaluated in terms of this research's aims and objectives, i.e. qualitative assessment of users perception and experience. 

\textbf{R10:} As with all other design processes, the DND method suggest that if the digital nudge did not successfully achieve the desired behavior, one should return to the design phase and make adjustments. However this was out of scope for this master thesis. 

\subsection{Content}
The digital nudge intervention was designed to provide health information related to physical activity. Because message framing was applied, it was natural to present positive health effects with physical activity and negative health effects with inactivity. The digital nudges are based on carefully selected information so as not to create commercial and controversial content that could overshadow the perception/experience. The topics were: muscle and skeleton, diabetes type 2, cancer, brain, life expectancy, endorphins and dopamine, mental health and cardiovascular. 

Applying information about health effects coheres with both the objectives of the training center, as they aim to promote physical activity, and the definition of nudging, that should support good choices for individuals best interest. As far as consensus research can gather, physical activity is in the best interest of the individual as well as the society. The presented knowledge is based on multiple reports and expenses from across the world and has been endorsed by international organizations like WHO and national organizations like FHI and Helsedirektoratet. The latter, currently has the leading role of informing Norwegian residents about physical activity. Basing the nudges on such information also contributes to making the information more available, which can be an important step towards reducing inactivity. 

\begin{figure}
    \includegraphics[width=1\textwidth]{images/Nudger.png}
    \caption{Overview of health information that was broadcasted in the digital nudge intervention.}\label{fig:my_label} 
\end{figure}

%\textbf{Tailoring}
\subsection{Tailoring}
%kanskje en del av introduction 
We are aware of the need for tailoring content, and that is also why we are testing this out. We want to gather insight on what should be taken into consideration when tailoring digital nudges. As the review implies, digital nudging is a comprehensive concept, consisting of many design aspects. As the research questions reflects we are approaching how the type of content (RQ1), framing effect (RQ2) and push notification as delivery method (RQ3), affects users perception and experience. 

%People who do not enjoy physical activity will justify the behaviour; Ved å presentere helseffekter kan det skape kognitiv dissonanse. Som vil si at for negative helseeffekter: kan det får en person til å trene fordi de har en intensjon om å ha en sunn helse og ønsker å unngå å ha høyere risiko for å bli syk. Positive helseeffekter: kan få dem som ikke liker å trene til å trene fordi den "rettferdiggjør" en atferd? 

\subsection{Randomization and broadcast plan}
The original plan was that 16 distinct nudges was randomly distributed over a period of 30 days. As timing were not a desired investigation objective, the broadcast of the nudges was randomized, but still within convenient time frames (morning 08-10, noon 12-15, evening 18-20). The original plan for broadcasting can be found in Appendix. Due to Covid-19 only 8 of the intended digital nudges was broadcasted. 

\subsection{Technical implementation}
The nudges were implemented in Sats' own developed CMS (content management system) that allows for publishing content through their app. The CMS communicates with their service platform delivered by Microsoft Azure. Further the service platform communicates with Google Firebase, which provides messaging through cloud, and makes it possible to send notifications to any operating system. Google Firebase connects with users Device ID (users smartphones) and delivers notifications to operating system on their devices. The nudges appeared as push notifications on users locked home screens, but were also presented as a in-app notification when opening the app. 

%"Selve push meldingen (den som operativsystemet gir bruker, og lever på utsiden av appen); Her bruker vi Google Firebase for å kommunisere med telefonen. Den tar utgangspunkt i Device ID (telefonen) og pusher til Device ID eller IDer (om brukeren er registert med flere devicer). Dette oppfattes som selve push-meldingen. Det er vår serviceplattform som kommuniserer med Google Firebase (og CMSet du kjenner til er det som pusher informasjon inn til vår service plattform). Vår service plattform er forresten ren Microsoft Azure."

\bigbreak
\bigbreak
\begin{figure}[ht]
\includegraphics[width=1\textwidth]{images/Nudge.png}
\caption{A delineation of the scope can be illustrated by a pseudo hierarchy that distinguish and sort the terms}
\end{figure}
\bigbreak
\bigbreak
