\chapter{Introduction} 
%\section{}
%\subsection{}
%\subsubsection{}

%Introduce your topic.• State of the world…• The big BUT…• Therefore, we did…• The key findings are…• The contributions of this work are…

\section{State of the art / Statement of problems}

The state of the world is characterized by two major facts right now; 1) we are surrounded by (and dependent of) technology everywhere, and 2) physical inactivity is a global health problem. These two put together, serves as a fruitful area of interest for Human-Computer Interaction (HCI) researches. 

Many of our day-to-day choices are made on a digital interfaces, such as mobile application and websites. Smart phones, laptops, tablets and other technological devices are everywhere around us, and present us with choices, whether we are at work, school, home or on the go. The unlimited access to screens and other technological widgets has made mankind dependent on, and in many cases succumbed to, such devices \cite{mirsch_digital_2017}.
%(Digital Nudging: Altering User Behavior in Digital Environments)
This also implies that this type of user interface is powerful and can influence people in certain directions if designed carefully. 

In addition, recommendations from policymakers and other stakeholders are not sufficient in motivating people for more physical activity. On a global level the lack of physical activity is still a problem. Also, it seems that to fill in this gap is very difficult. Most experts argue that creating such changes is more important than ever. Inactivity have become a big killer in our society \cite{kohl_pandemic_2012}.

Most people are aware of the many benefits physical activity gives us. In Norway, we learn about why physical activity is good for us already in the childhood, and it continues through school, education and work. However, health statistics indicate that education is not good enough to ensure people to stay physical active on a regular basis. Numbers from FHI, show that only 30 percent of Norway's population fulfill the minimum recommendations of physical activity \cite{folkehelseinstituttet_fysisk_nodate}. %(Reference: https://www.fhi.no/nettpub/ncd/fysisk-aktivitet/voksne/) 
A questionnaire where participants where supposed to self-asses their physical activity level, revealed that about 30 percent of adults (20-85 years) in Norway are defined as inactive. However, comparable data have been measured with an accelerometer shows that 70 percent fall inside same category. These data are of higher quality than self-reported data (from experience one can gather that individuals have tendency to exaggerate opinions and behaviour that would put them in a better light). The large gap between self-reported and measured physical activity may imply that people are less active than they appear, or believe in (self-deception). (It should be mentioned that the number of participants for the self-reporting survey was low, which gives uncertainty about the data being representative of the entire population.) 

Norway is one of many countries which has committed to WHO's global goal of reducing inactivity. The concrete goal for 2025, it to reduce it by 10 percent \cite{noauthor_inaktivitet_nodate}.
%(Reference: https://www.fhi.no/nyheter/2018/ny-handlingsplan-skal-fa-folk-opp-av-sofaen/). 
The Norwegian Institute of Public Health (FHI), a Norwegian governmental agency subject to the Ministry of Health and Care Services (Helsedirektoratet), is in demand of an overview of best practices and initiatives that contribute to physical activity among Norway's population groups. These practices and initiatives could evolve from technology and user interfaces.

Although the recommendations regarding physical activity from the Helsedirektoratet is available in rich detail online, they still have the potential to become more accessible and embedded in everyday tools and services, such as technological devices. (However, it is not that simple.) Making education and information more available and persuasive will not alone drive the people at the wrong side of the scale into becoming enough physical active. The psychology of human behavior and absence of behavior is more complex than our ignorance. Visiting the field of behavior science, B.J Fogg insist that there are several factors that needs to be present at the same time in order to achieve a target behaviour; motivation, ability and trigger/prompt \cite{fogg_persuasive_2003}, is supposed to be the key to success. 

Take the scenario of a training center: all members are seemingly have the ability and some level of motivation to be physical active as they are already paying the monthly fee. Now, here is the challenge, how do we trigger or prompt them? This is where technology comes in. It would be interesting to embed the knowledge from these stakeholders (FHI and Helsedirektoratet) into a digital context for people whoms motivation and ability is somewhat given. There are several ways to implement such an intervention. 
 
This trigger factor is where we can utilize technology and user interfaces. Further Fogg describe three different variations of the trigger; facilitator, spark and signal. The latter is best to use when motivation and ability is present, (where considerations regarding tailoring, frequency and timing) \cite{fogg_persuasive_2003}.

%There is a need to make this information more available and persuasive. 
%Therefore it is interesting to embed the knowledge from these actors into everyday technology (for instance like an gym app). 

So what kind of nudge/trigger/prompt should people be subject to, to change their behavior? This debate does not only revolve around making totally inactive people to become active, but ensuring those who are already active to continue on a regular basis. 

At the same time, we are aware of the 24 hour life, constant feed and the information overload. How do the technological and digital interventions survive in such environment? 

How do we reach the target audience with the message? What is needed for people to make use of a mobile application in general? The good user experiences. And in good user experiences there is often customized content. It has become an important design principle for making systems compelling, and can therefore be transferred to digital nudging as well. Since digital nudging is quite new and in shortage  qualitative research, we don't necessarily know how to tailor (it). In addition, digital nudging has not been implemented and studied in the fitness center context before. What applies to those users?

Among Norwegian citizens, 70 percent do not fulfill minimal recommendations of physical activity \cite{folkehelseinstituttet_fysisk_nodate}, meaning they are more prone to getting certain disease, illnesses and plagues, both physically and mentally, (than they need to be). The authorities and commercial players have a shared responsibility for ensuring good health among our population. Where to start? It seems logical to start a place where one is already on the right track (i.e. it is not about changing attitudes towards people as they are already motivated and aware of physical activity to a certain level). This study therefore approached the training center context, because here the users already have some of the prerequisites (motivation and ability) in order to achieve the targeted behaviour physical activity. Being member of a fitness center, does not necessarily mean that you meet the national requirements, at least not from week to week, because life happens, in short. This is therefore a suitable target group. In addition, there are many different people at a fitness center, which...?

%For it to be sustainable, it should also be taken into account that the changes must be long lasting and preferably permanent.

This study will transfer theories from the field of psychology into digital nudging, in order to gain insight on users perception and experience of the given implementation of digital nudging, which again can contribute to establishing if this is an effective method of doing it. 

We are investigating three factors; 1) how framing of message , 2) how the chosen technology channel, push notification, affects it and 3) how the content is perceived.  

%There has been done several studies regarding the impact of communication, framing and other. For instance in context of persuasive systems, health communication system, behaviour change system, etc. It can remind and sometimes overlap the concept of digital nudging. However, to the best of my knowledge this is the first experiment including framing for digital nudging in specific. 

%Intro: verdens bilde: mer teknologi + mindre aktivitet
%Snu det om, bruke teknologi til å få folk mer aktive
%Tidligere forskning, persuasive design og tech etc... hva har blitt gjort ang fysisk aktivitet
%Vi har forstått at vi må ta mer fra pyskologien for å knekke design koden å kunne lage de beeste profuktene
%det er dette hci mye bunner i
%emrging trends: digital nudging, som faktisk stammer fra behavoirual ecomics..
%litt hva det er og som er funnet
%her er også spørsmålet: how to nudge?
%må låne ting fra andre fagfelt, gjøre tester av ulike mekansimer og ways of implementation, for å kartlegge for hvilke situasjoenr man bør benytte seg av hvilke type digital nudging. 

This study performs an qualitative assessment of users perception of a implemented digital nudging intervention to promote physical activity. Aiming to gain insight and contribute with findings to the big overall question "how to nudge".

\section{Research Questions}
\begin{itemize}
\item RQ1: How are digital nudges based on consensus health information, experienced and perceived by gym members? 
\item RQ2: How does message framing affect the user perception of the digital nudge?
\item RQ3: How are push notifications experienced as the channel for broadcasting digital nudges? 
%\item RQ4: What should be taken into consideration for tailoring of digital nudges in this context? 
\end{itemize}

These questions are asked in order to answer the bigger problem statement: how to nudge and what to consider when tailoring digital nudges for physical activity promotion? Through this study I also aim at gaining insight into which factors should be considered when it comes to tailoring of digital nudges. The overall problem is thus how we should nudge, which implies that psychological effects should be exploited, how it should be implemented, etc. This question is very large and is therefore bounded by two more concrete questions; how framing and push notification come into play. The research questions is answered by qualitatively evaluating and investigating patterns, trends, tendencies, correlations. interview data based on users perceptions,  

\section{Aims}
The primary aim of the study is to understand user perceptions of the digital nudge intervention (taking message framing, push notification and content into consideration), in order to gather insight on how to tailor digital nudges and make them more effective in the given context. The gathered information will be used to strengthen and complement knowledge on nudging in general, and through digital environments such as mobile application. Such it will provide contributions to both academics, industry (commercial operators) and authorities. 

On the higher level, the aim would be to safeguard and improve public health by fostering good and healthy choices. This can be achieved by presenting and communicating health information and encourage the better choices through commonplace ubiquitous technology. This would be implemented using nudging mechanisms that build on psychological effects and processing. 

%multimodal interface 

tools and technologies. . and make it more available in commonplace 
 
 
 through technology and user interface design. 

%Making people life through new ways of applying HCI context, in set context health information, digital interface design
%Looking at how a certain implementation of digital nudging impacts attitudes towards regular physical activity by collecting qualitative data about users experience and perception

Current research on digital nudging is mainly emerging from the view of Information Systems (IS), and often refers to digital nudging as part of e-commerce and organizational. This study approach the research area from an HCI perspective as it concerns the bigger question "how to nudge" by investigating the design of a implemented digital nudge interface and tries to understand users' perceptions and experiences. HCI's role is to investigate this to find out how this can become part of existing user interfaces or individual solutions to promote physical activity. This is mainly done in order to establish how to get closer to the answer to "how to nudge" and how to achieve effective digital nudges. The goal is hence to contribute to a better understanding of digital nudging from a HCI perspective in the given context and users, and identify the area that needs to be studied more closely. 

To best of knowledge one of the first attempts 

\section{Objectives}
The objectives of the study were to design and perform a successful digital nudge implementation retrofit an already existing technology (mobile application), and rolling it out towards a selection of users.
Further, it was to conduct interviews and collect qualitative (and some quantitative) data about users perception and experience towards tested implementation, and evaluated in that manner. More specifically what concerned the framing, the technology channel, content, information and source credibility. It was an ambition of the study to follow a normative and accepted standard and honor the integrity of the participants. 

\section{Research design and context}
\subsection{Context  }
%This study uses a qualitative research design (answering the whys and hows), where data are collected through interviews. 

The study is conducted in 2020, in affiliation with the Masters of Human-Computer Interaction programme in the Department of technology at Kristiania University College. This master thesis is conducted in collaboration with Sats, one of the biggest fitness centers in Norway, to investigate the defined problem statement in a context with gym members. Sats have gym centres all over Scandinavia, which makes them a useful partner which offers an opportunity to include a wider target population across country borders. However, considering the scope of this master thesis, the study was limited to only including Norwegian members due to physical attendance for interviews (which later was restricted to be digital based due to Covid-19). 

Through this study we are conducting a digital nudging intervention in a real-life context with gym members of Sats. The nudges will be sent as a push notification through the app that Sats distributes to their members. The push notification will only include text and no other visual effect. The Sats app main functionality is membership identification (QR scan to enter gym centers), booking group sessions and log training. However, it can be categorized as a persuasive mobile application as they are implementing different elements in order to increase motivation for physical activity, and to make their members become regular trainers. Such elements is for instance: challenges, social comparison, feedback, etc. 

\subsection{Target population / study sample / participants}
%Aka: targeted users or population of interest.skal flyttes til intro? 
%Individer som oppfyller to av tre faktorer for å oppnå atferden, fysisk aktivitet. Ability and motivation. 

The target population for this study are individuals that have membership Sats. The study was carried out in the Oslo area in Norway.  

%Should I explain why they are the target population, or just describe the target population? 

Based on the prerequisites about presence of ability and motivation, the target population includes almost (except people with injuries that stops them from engaging in physical activity or those who pay for the membership without any kind of motivation) everyone who is a member of a training center in Norway, which is a rather large group of people. It would be demanding to reach out for all of them, if not impossible. Therefore, there was made some frames and limitations in order to form a sample of the population. The sample population for this study consists of people with membership at Sats and that has downloaded their membership-app. 
 
The study sample is representative to its target population to some degree, as the participants varies in gender, age, motivation type, prior frequency of physical activity, attitudes, etc. (However, they do all belong to the same geographical area, as this was one of the frames for the sample).




