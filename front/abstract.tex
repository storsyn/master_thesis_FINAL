\chapter*{Abstract}
The utilization of technology and design, are important tools for making people take good healthy choices. Persuasive technologies have been widely explored for physical activity promotion. A new concept, digital nudging, has emerged from behavioral economics and refers to the intentional interface design to foster people's decision-making regarding real-world behaviors. How to nudge in digital environments needs more investigation. Due to its complexity and novelty, it is a promising field for HCI research. A full scale digital nudge intervention was set up in collaboration with an major training center. Digital nudges were broadcasted through an already existing platform (training app). The nudges were based on consensus health information that pertains to the effects of being physically active. Post the digital intervention, user data was collected from the targeted user group. The data became subject to a qualitative review and assessment. Research questions and themes were defined accordingly. Qualitative assessment hinged around user experiences, effect of framing and suitability of delivery method.The findings indicate that the intervention had a small positive effect on the majority of participants, and a strong negative effect on a smaller group of participants. The delivery method by far succeeded in bringing the nudges to attention, Participants appreciated the low level of commitment and unconditional communications was favoured. Statement are also made about educational effects and urgency hierarchy. 

\textbf{Keywords:} digital nudging, persuasive technology, health information, message framing, push notification, user experience, HCI, training center, qualitative assessment